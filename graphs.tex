\documentclass[a4paper]{article}
\usepackage[utf8]{inputenc}
\usepackage[T2A]{fontenc}
\usepackage{amsfonts}
\usepackage{amsmath, amsthm}
\usepackage{amssymb}
\usepackage{hyperref}
\usepackage{multicol}

\newcommand\letsymbol{\mathord{\sqsupset}}
\usepackage[russian]{babel}
\renewcommand\qedsymbol{$\blacktriangleright$}
\newtheorem{theorem}{Теорема}[section]
\newtheorem{lemma}{Лемма}[section]
\theoremstyle{definition}
\newtheorem*{example}{Пример}
\newtheorem*{definition}{Определение}
\theoremstyle{remark}
\newtheorem*{remark}{Замечание}

\setlength{\topmargin}{-0.5in}
\setlength{\oddsidemargin}{-0.5in}
\textwidth 185mm
\textheight 250mm

\begin{document}
    \tableofcontents
    \pagenumbering{arabic}
    \setcounter{page}{1}
    \section{Определение графа. Примеры графов. Степени вершин графа. Лемма о рукопожатиях}
    \section{Маршруты, цепи, циклы. Лемма о выделении простой цепи. Лемма об объединении 
    простых цепей}
    \section{Эйлеровы графы. Критерий существования эйлерова цикла (теорема Эйлера)}
    \section{Гамильтоновы графы. Достаточные условия существования гамильтонова цикла (теоремы 
    Оре и Дирака)}
    \section{Изоморфизм графов. Помеченные и непомеченные графы. Теорема о числе помеченных 
    n-вершинных графов}
    \section{Проблема изоморфизма. Инварианты графа. Примеры инвариантов. Пример полного 
    инварианта}
    \section{Связные и несвязные графы. Лемма об удалении ребра. Оценки числа ребер связного 
    графа}
    \section{Плоские и планарные графы. Графы Куратовского. Формула Эйлера для плоских графов}
    \section{Деревья. Теорема о деревьях (критерии)}
    \section{Перечисление деревьев. Теорема Кэли о числе помеченных n-вершинных деревьев}
    \section{Центр дерева. Центральные и бицентральные деревья. Теорема Жордана}
    \section{Изоморфизм деревьев. Процедура кортежирования. Теорема Эдмондса}
    \section{Вершинная и реберная связность графа. Основное неравенство связности}
    \section{Отделимость и соединимость. Теорема Менгера}
    \section{Реберный вариант теоремы Менгера}
    \section{Критерии вершинной и реберной k-связности графа (без доказательства)}
    \section{Ориентированные графы. Основные понятия. Ормаршруты и полумаршруты. 
    Ориентированые аналоги теоремы Менгера}
    \section{Ориентированные графы. Достижимость и связность. Три типа связности. Критерии 
    сильной, односторонней и слабой связности орграфа}
    \section{ Основные структуры данных для представления графов в памяти компьютера. Их 
    достоинства и недостатки}
    \section{Влияние структур данных на трудоемкость алгоритмов (на примере алгоритма 
    отыскания эйлерова цикла)}
    \section{Задача о минимальном остовном дереве. Алгоритм Прима}
    \section{Задача о кратчайших путях. Случай неотрицательных весов дуг. Алгоритм Дейкстры}
    \section{Потоки в сетях. Увеличивающие пути. Лемма об увеличении потока}
    \section{Алгоритм Эдмондса-Карпа построения максимального потока}
    \section{Разрезы. Лемма о потоках и разрезах. Следствие}
    \section{Теорема Форда-Фалкерсона}
    \section{Два критерия максимальности потока.}
    \section{Приложения теории потоков в сетях. Задачи анализа структурно-надежных 
    коммуникационных сетей}
    \section{Задачи комбинаторной оптимизации. Массовая и индивидуальная задачи. 
    Трудоемкость алгоритма. Полиномиальные и экспоненциальные алгоритмы}
    \section{Задачи распознавания свойств. Детерминированные и недетерминированные 
    алгоритмы. Классы P и NP. Проблема “P vs NP”}
    \section{Полиномиальная сводимость задач распознавания. Свойства полиномиальной 
    сводимости}
    \section{NP-полные задачи распознавания. Теорема о сложности NP-полных задач. Примеры 
    NP-полных задач}
\end{document}