\documentclass[a4paper, 12pt]{article}
\usepackage[utf8]{inputenc}
\usepackage[T2A]{fontenc}
\usepackage{amsfonts}
\usepackage{amsmath, amsthm}
\usepackage{amssymb}
\usepackage{hyperref}
% \usepackage{mathrm}
\usepackage{bbold}
\usepackage{mathrsfs}


\usepackage[russian]{babel}

\renewcommand\qedsymbol{$\blacktriangleright$}
\newtheorem{theorem}{Теорема}[section]
\newtheorem{lemma}{Лемма}[section]
\theoremstyle{definition}
\newtheorem{example}{Пример}
\newtheorem{problem}{Упражнение}[section]
\newtheorem*{definition}{Определение}
\theoremstyle{remark}
\newtheorem*{remark}{Замечание}


\DeclareMathOperator{\Int}{int}
\DeclareMathOperator{\clo}{cl}

\newcommand\letsymbol{\mathord{\sqsupset}}

\setlength{\topmargin}{-0.5in}
\setlength{\oddsidemargin}{-0.5in}
\textwidth 185mm
\textheight 250mm

\begin{document}
Общее понятие топологического векторного пространства и несколько
более специальное понятие локально выпуклого пространства были введены
А.Н. Колмогоровым, А.Н. Тихоновым и Дж. фон Нейманом в 1934-1935 гг., 
к этому же времени относятся первые результаты о свойствах таких пространств.
Наиболее интенсивное развитие теории топологических векторных пространств началось
в конце 1940 - начале 1950 годов и во многом было связано с созданием Л. Шварцем теории
распределений (обобщенных функций), в рамках которой был выделен широкий спектр важных для приложений
функциональных пространств, являющихся общими топологическими векторными пространствами.

Излагаемый здесь материал призван помочь студентам-математикам в изучении элементов
теории топологических векторных пространств - одного из важнейших разделов функционального анализа и состоит из
заданий, упражнений и методических указаний по их исполнению. Задания и упражнения приводятся в порядке,
позволяющем без особых затруднений выполнять последующие на основе предыдущих. Настоящее
издание является продолжением методических указаний автора ''Элементы общей топологии''
и ''Векторные пространства'', поэтому желательно знакомство читателя с их содержанием.
Если при выполнении заданий и упражнений возникнут трудности, то следует обратиться к литературе,
список которой указан на с.41. Для удобства ссылок этот список открывают вышеупомянутые методические указания.

\section{Топологические векторные пространства}
% \chaptermark{ТВП}

Пусть $X$ - векторное пространство над полем $\mathbb{K}$ вещественных или комплексных чисел.
В дальнейшем предполагается, что $\mathbb{K}$ наделено естественной топологией.
Топология $\tau$ на Х согласована с линейной (векторной) структурой Х, 
если непрерывны линейные операции, т.е. операции сложения векторов и умножения векторов
на скаляры:
\[(+): X\times X \ni (x, y) \longmapsto x+y \in X,\]
\[(\cdot ): \mathbb{K} \times X \ni (\lambda, x)\longmapsto \lambda x \in X.\]
Согласованная с линейной структурой топология на Х называется линейной топологией.
Векторное пространство Х с заданной на нем линейной топологией называется топологическим
векторным пространством (ТВП). Наравне с этим термином бытует также термин ''линейное топологическое пространство'' (ЛТП).

Через $\mathcal{O} _x^X, \mathcal{O} _x^\tau$ или $\mathcal{O} _x$ обозначаем множество
всех окрестностей точки х в топологическом (векторном) пространстве $(X, \tau)$.
Если $x = 0$, то пишем просто $\mathcal{O} ^X, \mathcal{O}^\tau$ или $\mathcal{O} $.
Через $\Int A$ и $\clo A$ обозначаем соответственно внутренность и замыкание множества
$A\subset X$.

В дальнейшем неоднократно будет использоваться понятие (пред) фильтра, поэтому напомним его.
Непустое множество $\mathcal{U} $ подмножеств $X$ называется предфильтром в Х, если $\varnothing\in \mathcal{U} $
и для любых $A, B \in \mathcal{U} $ существует такое $C\in \mathcal{U} $, что $C\in A\cap B$.
Предфильтр $\mathcal{F}$ называется фильтром, если $B\in \mathcal{F}$, как только $B \supset A \in \mathcal{F}$.
Предфильтр $\mathcal{U}$ является базой (базисом) фильтра $\mathcal{F}$, если $\mathcal{F}$,
если $U\subset \mathcal{F}$ и $\forall B\; \exists A\in \mathcal{U}: \quad A\subset B$. Каждый предфильтр
$\mathcal{U}$ является базой порожденного им фильтра $\overline{\mathcal{U}} = \{B\subset X \;|\; \exists A \in \mathcal{U}: \; A\subset B\}$.
Наиболее распространенным примером фильтра является фильтр $\mathcal{O}^X _x$ окрестностей
точки $x$ в топологическом пространстве $X$.
\begin{problem}
    Сформулировать условия линейности топологии на векторном пространстве в терминах окрестностей.
\end{problem}
\begin{problem}
    Исследовать на линейность дискретную и антидискретную топологии на векторном пространстве.
\end{problem}
\begin{problem}
    Доказать, что векторное пространство $\mathbb{K}^n$ с естественной топологией, т.е. топологией,
    определяемой окрестностями 
    \[V_r(x) = \{y\in \mathbb{K}^n:\; (y_1-x_1)^2 + \dots + (y_n -x_n)^2 < r^2\}, \; r >0, \; x\in \mathbb{K}^n,\]
    является топологическим векторным пространством. В дальнейшем $\mathbb{K}^n$ всегда
    предполагается наделенным этой топологией.
\end{problem}
\begin{problem}
    В векторном пространстве $C(\mathbb{R})$ каждой функции $x\in C(\mathbb{R})$ сопоставим семейство
    ее окрестностей:
    \[V_r(x) = \{y\in C(\mathbb{R}):\; \sup_{t\in\mathbb{R}}|y(t)-x(t)|<r \}, \; r >0.\]
    Доказать, что базы окрестностей $\{V_r(x)\;|\;r>0\}\; (x\in C(\mathbb{R}))$ определяют на $C(\mathbb{R})$
    топологию (см. I.I.13[I]), при которой операция сложения непрерывна, а операция умножения на скаляры
    разрывна в каждой точке $(\lambda, x)\in \mathbb{K}\times C(\mathbb{R})$, где $x$ - неограниченная функция.
    Таким образом, указанная топология не является линейной на $C(\mathbb{R})$
\end{problem}
\begin{problem}
    Привести пример линейной топологии на $C(\mathbb{R})$.
\end{problem}
\begin{problem}
    На пространстве $C[a, b]$ топологию введем с помощью окрестностей
    \[V_r(x) = \{y\in C[a, b] : \ \sup_{a \le t \le b} |y(t) - x(t)| < r\}, r > 0.\]
    Доказать, что эта топология - линейная.
\end{problem}
\begin{problem}
    На пространстве $l^p \ (0 < p < \infty)$ топологию зададим посредством окрестностей
    \[V_r(x) = \{y = \{y_n\}\in l^p \ : \ \sum_{n=1}^{\infty}|y_n-x_n|^p< r\}, r>0.\]
    Доказать, что эта топология - линейная.
\end{problem}
\begin{problem}
    Топология на $\mathscr{L}^p (0, 1) \ (0 < p < \infty)$ определяется окрестностями
    \[V_r(x) = \{y\in \mathscr{L}^p : \ \int_0^1 |y(t) - x(t)|^p dt < r\}, r > 0.\]
    Доказать, что эта топология - линейная.
\end{problem}
\begin{problem}
    На пространстве $C(\mathbb{R})$ топологию определим с помощью окрестностей
    \[V_{K, r}(x) = \{y\in C(\mathbb{R})\ : \ \sup_{t\in K} |y(t) - x(t)|< r\},\]
    где $r > 0$. $K$ - компакт в $\mathbb{R}$.
    Доказать, что эта топология - линейная.
\end{problem}
\begin{remark}
    Всюду, где не оговорено, пространства $C[a, b], l^p, \mathscr{L}^p (0 < p < \infty), C(\mathbb{R})$
    предполагаются наделенными указанными выше линейными топологиями.
\end{remark}
% TODO a б в... 
\begin{problem}
    Пусть $X$ - ТВП. Тогда:
    \begin{enumerate}
        \item для любых $x\in X, \lambda \in \mathbb{K}\setminus \{0\}$
        отображение $y\longmapsto \lambda y + x$ является гомеоморфизмом
        $X$ на себя;
        \item $\forall \ \lambda \in \mathbb{K}\setminus \{0\} \ (V\in \mathcal{O} \implies \lambda V\in \mathcal{O})$;
        \item $\forall \ x \in X \ \mathcal{O}_x = \{x + V \ | \ V\in \mathcal{O}\}, \quad \mathcal{O} = \{x - V \ | \ V\in \mathcal{O}_x\}$.
    \end{enumerate}
\end{problem}
\begin{problem}
    Пусть $X$ - ТВП. Тогда:
    \begin{enumerate}
        \item $\forall \ V\in \mathcal{O} \ \exists \ U\in \mathcal{O} \ : \ U + U \subset V$;
        \item $\forall \ V \in \mathcal{O} \ \forall \ n \in \mathbb{N} \ \exists \ U \in \mathcal{O} \ : \ \underbrace{U + \cdots + U}_n \subset V$;
        \item $\mathcal{O}$ состоит из поглощающих множеств, в частности,
        \[\forall \ V \in \mathcal{O} \ \forall \ \{r_n\} \subset (0, +\infty) \ : \ \lim_{n\to \infty} r_n = +\infty \quad X = \bigcup_{n\ge 1} r_n V;\]
        \item в каждой окрестности нуля содержится уравновешенная окрестность нуля.
        

    \end{enumerate}
    Указание. Для доказательства в) (соответственно г)) воспользуйтесь непрерывностью произведения на скаляры в точках $(0, x)\in \mathbb{K}\times X$ (соотв. в точке $(0, \mathbb{0})$).
\end{problem}
Говорят, что топология на векторном пространстве $X$ инвариантна относительно 
сдвигов, если все сдвиги на $X$ суть гомеоморфизмы.
\begin{problem}
    Для линейности топологии $\tau$ на $X$ необходимо и достаточно, чтобы
    она она была инвариантна относительно сдвигов и имела базис 
    окрестностей нуля $\mathcal{U}$, удовлетворяющий условиям:
    \begin{enumerate}
        \item $\forall \; V \in \mathcal U \ \exists \; U \in \mathcal U \ : \ U + U \subset V$;
        \item $\mathcal U$ состоит из поглощающих уравновешенных множеств.
    \end{enumerate}
    Указание. При доказательстве непрерывности произведения на скаляры удобно воспользоваться представлением 
    \[\lambda x = \lambda_0 x_0 + (\lambda - \lambda_0) x_0 + \lambda (x-x_0).\]
\end{problem}
\begin{problem}
    Если предфильтр $\mathcal U$ на векторном пространстве $X$ удовлетворяет условиям:
    \begin{enumerate}
        \item $\forall \ V\in \mathcal U \ \exists \; U \in \mathcal U \ : \ U + U \subset V$;
        \item $\mathcal U$ состоит из поглощающих уравновешенных множеств,
    \end{enumerate}
    то на $X$ существует единственная линейная топология $\tau$, для которой $\mathcal U$ - база окрестностей нуля.

    Указание. Положите $\tau = \{V \subset X \; |\; \forall \; x\in V \ \exists \; U \in \mathcal U \ : \ x + U \subset V\}$.

    Докажите, что $\forall \ W \in \mathcal U \ W_0 := \{x\in W \ | \ \exists \; V \in \mathcal U \ : \ x + V \subset W\} \in \tau$, для чего проверьте, что если $x\in W_0,\; x\in V \subset W, \ U + U \subset V \ (U, V \in \mathcal U)$, то $x + U \subset W_0$. Покажите, что $\mathcal U$ - база $\mathcal O^\tau$ и топология $\tau$ инвариантна относительно сдвигов.
\end{problem}
\begin{problem}
    Каждая окрестность нуля в ТВН содержит некоторую замкнутую уравновешенную окрестность нуля. В частности, каждое ТВП является $T_3$-пространством (см. I.3.3[I]).

    Указание. Воспользуйтесь свойствами а), б) из I.12 и тем, что $(x-U)\;\cap \; U \neq \varnothing$ при $U\in \mathcal U, x\in \clo U$.
\end{problem}
\begin{problem}
    Пусть $X$ - ТВП. $A, B \subset X$. Тогда:
    \begin{enumerate}
        \item если $A$ открыто, то $\lambda A + B$ открыто для любого $\lambda \in \mathbb{K} \setminus \{0\}$;
        \item $\clo A = \bigcap \{A + V \; | \; V\in \mathcal U\}$ для любой базы $\mathcal U$ фильтра $\mathcal O$;
        \item $\clo A + \clo B \subset \clo (A+B)$;
        \item $\Int A + \Int B \subset \Int (A+B)$;
        \item если $A$ уравновешено (соотв., является подпространством $X$), то $\clo A$, а если $\mathbb o \in \Int A$, то и $\Int A$ уравновешено (соотв., является подпространством $X$);
        \item если $A$ выпукло, то $\clo A$ и $\Int A$ выпуклы;
        \item если $A$ выпукло и $\Int A\neq \varnothing$, то:
        \[\clo (\Int A) = \clo A, \quad \Int (\clo A) = \Int A;\]
        \item если $A$ - подпространство $X$ и $\Int A\neq \varnothing$, то $A = X$.
    \end{enumerate}
    Указание. а) вытекает из 1.10; доказательства остальных утверждений основаны на том, что замыкание $\clo A$ множества $A$ есть наименьшее замкнутое множество, содержащее $А$, а его внутренность $\Int A$ - наибольшее открытое подмножество $A$.

    Пример. Множество $A = \{(x, y)\in \mathbb{R}^2 : |x|\le |y|\}$ уравновешено в ТВП $\mathbb{R}^2$, то его внутренность не является уравновешенным множеством.

    Вопрос: справедливы ли утверждения д)-ж) для произвольного $\Gamma$-множества $A$ (см. [2, с.26]) ?
\end{problem}
Топологическое векторное пространство называется локально
(полу)выпуклым, если в нем каждая окрестность нуля содержит (полу)выпуклую
окрестность нуля или, что то же самое, существует
база окрестностей нуля, состоящая из полу(выпуклых) множеств.
Его топология называется локально (полу)выпуклой. Напомним,
что множество $V$ полувыпукло, если $V+V\subset \lambda V$ для
некоторого $\lambda > 0$. Локально выпуклое топологическое
векторное пространство обычно называют просто
локально выпуклым пространством (ЛВП).
\begin{problem}
 Какие пространства из упражнений 1.3, 1.6 - 1.9 являются локально выпуклыми?
\end{problem}
\begin{example}
    При $0 < p < 1$ в пространстве $\mathcal {L}^p (0,1)$ нет выпуклых открытых множеств, отличных от $\varnothing$ и $\mathcal {L}^p (0,1)$. В частности, $\mathcal {L}^p (0,1) (0< p < 1)$ не является локально выпуклым пространством.

    Действительно, пусть $V$ - непустое выпуклое открытое множество в $\mathcal {L}^p$. Можно считать, что $\mathbb{o}\in V$. Тогда $V\supset V_r(\mathbb{o})$ для некоторого $r > 0$. Взяв $x\in \mathcal {L}^p $, найдем $n\in \mathbb{N}$ такое, что $n^{p-1}N(x)< r$, где $N(x) = \int_0^1 |x(t)|dt$. Положим $x_i(t) = nx(t)$ при $t_{i-1}< t \le t_i$ и $x_i(t) = 0$ в противном случае, где точки $0 = t_0< t_1 < \dots < t_n = 1$ таковы, что
    \[\int_{t_{i-1}}^{t_i} |x(t)|^p dt = n^{-1} N(x) \quad (1\le i \le n).\]
    Так как $N(x_i)< r$, то $x_i\in V (1 \le i \le n)$ и $x = \frac{1}{n} (x_1 + \dots + x_n)\in V$. Таким образом, $V = \mathcal L ^p$
\end{example}
\begin{problem}
    Если множество $V$ в ТВП $X$ открыто или замкнуто, то для его выпуклости достаточно выполнения равенства $V+V=2V$.

    Указание. Согласно условию, множество $V$ вместе с каждыми двумя своими точками содержит середину соединяющего их отрезка. В случае, тогда V открыто, удобно воспользоваться 1.20.
\end{problem}
\begin{problem}
    Пусть $A$ и $B$ - подмножества ТВП. Тогда:
    \begin{enumerate}
        \item Если $A$ и $B$ бикомпактны, то $\lambda A + \mu B$ бикомпактно для $\lambda, \mu \in \mathbb{K}$;
    \end{enumerate}
\end{problem}
\begin{problem}
    
\end{problem}
\begin{problem}
    
\end{problem}
\begin{problem}
    
\end{problem}
\begin{problem}
    
\end{problem}
\end{document}