\documentclass[a4paper, 12pt]{article}
\usepackage[utf8]{inputenc}
\usepackage[T2A]{fontenc}
\usepackage{amsfonts}
\usepackage{amsmath, amsthm}
\usepackage{amssymb}
\usepackage{hyperref}
\usepackage{mathrsfs}


\usepackage[russian]{babel}

\renewcommand\qedsymbol{$\blacktriangleright$}
\newtheorem{theorem}{Теорема}[section]
\newtheorem{lemma}{Лемма}[section]
\theoremstyle{definition}
\newtheorem*{example}{Пример}
\newtheorem{problem}{Упражнение}[section]
\newtheorem*{definition}{Определение}
\theoremstyle{remark}
\newtheorem*{remark}{Замечание}


\DeclareMathOperator{\Int}{int}
\DeclareMathOperator{\clo}{cl}

\newcommand\letsymbol{\mathord{\sqsupset}}

\setlength{\topmargin}{-0.5in}
\setlength{\oddsidemargin}{-0.5in}
\textwidth 185mm
\textheight 250mm

\begin{document}
Общее понятие топологического векторного пространства и несколько
более специальное понятие локально выпуклого пространства были введены
А.Н. Колмогоровым, А.Н. Тихоновым и Дж. фон Нейманом в 1934-1935 гг., 
к этому же времени относятся первые результаты о свойствах таких пространств.
Наиболее интенсивное развитие теории топологических векторных пространств началось
в конце 1940 - начале 1950 годов и во многом было связано с созданием Л. Шварцем теории
распределений (обобщенных функций), в рамках которой был выделен широкий спектр важных для приложений
функциональных пространств, являющихся общими топологическими векторными пространствами.

Излагаемый здесь материал призван помочь студентам-математикам в изучении элементов
теории топологических векторных пространств - одного из важнейших разделов функционального анализа и состоит из
заданий, упражнений и методических указаний по их исполнению. Задания и упражнения приводятся в порядке,
позволяющем без особых затруднений выполнять последующие на основе предыдущих. Настоящее
издание является продолжением методических указаний автора ''Элементы общей топологии''
и ''Векторные пространства'', поэтому желательно знакомство читателя с их содержанием.
Если при выполнении заданий и упражнений возникнут трудности, то следует обратиться к литературе,
список которой указан на с.41. Для удобства ссылок этот список открывают вышеупомянутые методические указания.

\section{Топологические векторные пространства}
% \chaptermark{ТВП}

Пусть $X$ - векторное пространство над полем $\mathbb{K}$ вещественных или комплексных чисел.
В дальнейшем предполагается, что $\mathbb{K}$ наделено естественной топологией.
Топология $\tau$ на Х согласована с линейной (векторной) структурой Х, 
если непрерывны линейные операции, т.е. операции сложения векторов и умножения векторов
на скаляры:
\[(+): X\times X \ni (x, y) \longmapsto x+y \in X,\]
\[(\cdot ): \mathbb{K} \times X \ni (\lambda, x)\longmapsto \lambda x \in X.\]
Согласованная с линейной структурой топология на Х называется линейной топологией.
Векторное пространство Х с заданной на нем линейной топологией называется топологическим
векторным пространством (ТВП). Наравне с этим термином бытует также термин ''линейное топологическое пространство'' (ЛТП).

Через $\mathcal{O} _x^X, \mathcal{O} _x^\tau$ или $\mathcal{O} _x$ обозначаем множество
всех окрестностей точки х в топологическом (векторном) пространстве $(X, \tau)$.
Если $x = 0$, то пишем просто $\mathcal{O} ^X, \mathcal{O}^\tau$ или $\mathcal{O} $.
Через $\Int A$ и $\clo A$ обозначаем соответственно внутренность и замыкание множества
$A\subset X$.

В дальнейшем неоднократно будет использоваться понятие (пред) фильтра, поэтому напомним его.
Непустое множество $\mathcal{U} $ подмножеств $X$ называется предфильтром в Х, если $\varnothing\in \mathcal{U} $
и для любых $A, B \in \mathcal{U} $ существует такое $C\in \mathcal{U} $, что $C\in A\cap B$.
Предфильтр $\mathcal{F}$ называется фильтром, если $B\in \mathcal{F}$, как только $B \supset A \in \mathcal{F}$.
Предфильтр $\mathcal{U}$ является базой (базисом) фильтра $\mathcal{F}$, если $\mathcal{F}$,
если $U\subset \mathcal{F}$ и $\forall B\; \exists A\in \mathcal{U}: \quad A\subset B$. Каждый предфильтр
$\mathcal{U}$ является базой порожденного им фильтра $\overline{\mathcal{U}} = \{B\subset X \;|\; \exists A \in \mathcal{U}: \; A\subset B\}$.
Наиболее распространенным примером фильтра является фильтр $\mathcal{O}^X _x$ окрестностей
точки $x$ в топологическом пространстве $X$.
\begin{problem}
    Сформулировать условия линейности топологии на векторном пространстве в терминах окрестностей.
\end{problem}
\begin{problem}
    Исследовать на линейность дискретную и антидискретную топологии на векторном пространстве.
\end{problem}
\begin{problem}
    Доказать, что векторное пространство $\mathbb{K}^n$ с естественной топологией, т.е. топологией,
    определяемой окрестностями 
    \[V_r(x) = \{y\in \mathbb{K}^n:\; (y_1-x_1)^2 + \dots + (y_n -x_n)^2 < r^2\}, \; r >0, \; x\in \mathbb{K}^n,\]
    является топологическим векторным пространством. В дальнейшем $\mathbb{K}^n$ всегда
    предполагается наделенным этой топологией.
\end{problem}
\begin{problem}
    В векторном пространстве $C(\mathbb{R})$ каждой функции $x\in C(\mathbb{R})$ сопоставим семейство
    ее окрестностей:
    \[V_r(x) = \{y\in C(\mathbb{R}):\; \sup_{t\in\mathbb{R}}|y(t)-x(t)|<r \}, \; r >0.\]
    Доказать, что базы окрестностей $\{V_r(x)\;|\;r>0\}\; (x\in C(\mathbb{R}))$ определяют на $C(\mathbb{R})$
    топологию (см. I.I.I3[I]), при которой операция сложения непрерывна, а операция умножения на скаляры
    разрывна в каждой точке $(\lambda, x)\in \mathbb{K}\times C(\mathbb{R})$, где $x$ - неограниченная функция.
    Таким образом, указанная топология не является линейной на $C(\mathbb{R})$
\end{problem}
\begin{problem}
    Привести пример линейной топологии на $C(\mathbb{R})$
\end{problem}
\begin{problem}
    На пространстве $C[a, b]$ топологию введем с помощью окрестностей
    \[V_r(x) = \{y\in C[a, b] : \ \sup_{a \le t \le b} |y(t) - x(t)| < r\}, r > 0.\]
    Доказать, что эта топология - линейная.
\end{problem}
\begin{problem}
    На пространстве $l^p \ (0 < p < \infty)$ топологию зададим посредством окрестностей
    \[V_r(x) = \{y = \{y_n\}\in l^p \ : \ \sum_{n=1}^{\infty}|y_n-x_n|^p< r\}, r>0.\]
    Доказать, что эта топология - линейная.
\end{problem}
\begin{problem}
    Топология на $\mathscr{L}^p (0, 1) \ (0 < p < \infty)$ определяется окрестностями
    \[V_r(x) = \{y\in \mathscr{L}^p : \ \int_0^1 |y(t) - x(t)|^p dt < r\}, r > 0.\]
    Доказать, что эта топология - линейная.
\end{problem}
\begin{problem}
    Доказать, что эта топология - линейная.
\end{problem}
\begin{remark}
    Всюду, где не оговорено, пространства $C[a, b], l^p, \mathscr{L}^p (0 < p < \infty), C(\mathbb{R})$
    предполагаются наделенными указанными выше линейными топологиями.
\end{remark}
\begin{problem}
    
\end{problem}
\begin{problem}
    
\end{problem}
\end{document}