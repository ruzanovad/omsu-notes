\documentclass[a4paper]{article}
\usepackage[utf8]{inputenc}
\usepackage[T2A]{fontenc}
\usepackage{amsfonts}
\usepackage{amsmath, amsthm}
\usepackage{amssymb}
\usepackage{hyperref}
\usepackage{multicol}
\usepackage{tikz} 

\newcommand\letsymbol{\mathord{\sqsupset}}
\usepackage[russian]{babel}
\renewcommand\qedsymbol{$\blacktriangleright$}
\newtheorem{theorem}{Теорема}[section]
\newtheorem{lemma}{Лемма}[section]
\theoremstyle{definition}
\newtheorem*{example}{Пример}
\newtheorem*{definition}{Определение}
\theoremstyle{remark}
\newtheorem*{remark}{Замечание}
\newtheorem*{corollary}{Следствие}

\setlength{\topmargin}{-0.5in}
\setlength{\oddsidemargin}{-0.5in}
\textwidth 185mm
\textheight 250mm

\begin{document}
    \tableofcontents
    \section{Интегралы, зависящие от параметра}
    \subsection{	Интегралы зависящие от параметра. Принцип равномерной сходимости}
    \subsection{	Теорема о коммутировании двух предельных переходов. Предельный переход под  знаком интеграла}
    \subsection{	Теорема о непрерывности интеграла, зависящего от параметра}
    \subsection{	Дифференцирование под знаком интеграла. Правило Лейбница}
    \subsection{	Интегрирование под знаком интеграла}
    \subsection{	Непрерывность и дифференцируемость интеграла с переменными пределами интегрирования}
    \subsection{	Равномерная сходимость интегралов. Достаточные признаки равномерной сходимости}
    \subsection{	Предельный переход в несобственном интеграле, зависящем от параметра}
    \subsection{	Дифференцирование  по параметру несобственного интеграла}
    \subsection{	Интегрирование по параметру несобственного интеграла}
    
    \section{Кратные интегралы}
    \subsection{ Двоичные разбиения. Двоичные интервалы, полуинтревалы, кубы. Свойства двоичных инервалов, кубов}
    \subsection{ Ступенчатые функции. Интеграл от ступенчатой функции (естественное и индуктивное определения). Теорема о совпадении определений}
    \subsection{ Свойства интеграла от ступенчатой функции (линейность интеграла, положительность, оценка интеграла)}
    \subsection{ Теорема о пределе интегралов убывающей последовательности функций, поточечно сходящейся к нулю}
    \subsection{ Теорема о пределе интегралов убывающей последовательности ступенчатых функций, поточечно сходящейся к нулю}
    \subsection{ Системы с интегрированием. Основной пример. Свойства систем с интегрирование}
    \subsection{ L1 норма. Множество L1*($\Sigma$). L1-норма как интеграл от модуля функции}
    \subsection{ Свойства L1 нормы ("линейность", норма функции равной нулю почти всюду и т.д.)}
    \subsection{ Субаддитивность L1-нормы}
    \subsection{ Сходимость в смысле L1}
    \subsection{ Определение понятие интеграла и интегрируемой функции}
    \subsection{ Свойства интеграла и интегрируемых функций}
    \subsection{ Множества меры ноль. Свойства функций совпадающих почти всюду}
    \subsection{ Нормально сходящиеся ряды. Теорема о нормально сходящихся рядах}
    \subsection{ Теоремы Леви для функциональных рядов и последовательностей}
    \subsection{ Огибающие для последовательности интегрируемых функций. Нижний и верхний предел последовательности}
    \subsection{ Теорема Фату о предельном переходе. Следствие из теоермы Фату}
    \subsection{ Теорема Лебега о предельном переходе}
    \subsection{ Лемма о приближении стпенчатой функции с помощью непрерывных финитных}
    \subsection{ Теорема о приближении интегрируемой функции с помощью непрерывных финитных}
    \subsection{ Измеримые функции. Свойства пространства измеримых функций. Измеримые множества}
    \subsection{ Теорема об интегрируемости измеримой функции}
    \subsection{ Теорема об измеримости предела измеримых функций}
    \subsection{ Теорема об интегрируемости предела возрастающей
    последовательности положительных измеримых функций}
    \subsection{ Обобщенно измеримые функции. Измеримые множества, мера множества. Теорема об измеримости объединения и пересечения измеримых множеств}
    \subsection{ Счетная аддитивность интеграла и меры}
    \subsection{ Измеримые множества в Rn. Внешняя мера множества. Лемма о представлении открытого множества как объединения кубов. Теорема об измеримости открытых и замкнутых множеств в Rn}
    \subsection{ Теорема о внешней мере множества}
    \subsection{ Лемма о приближении неотрицательной вещественной функции ступенчатыми функциями. Следствие об измеримости непрерывной почти всюду функции}
    \subsection{ Теорема о совпадении интералов Римана и Лебега}
    \subsection{ Теорема Фубини и следствия из нее}
    \subsection{ Теорема Тонелли и следствия из нее}
    \subsection{ Диффеоморфизмы и их свойства. Теорема о замене переменной в кратном интеграле (формулировка)}
    \subsection{ Лемма о замене переменной при композиции диффеоморфизмов}
    \subsection{ Лемма о сведении замены переменной в общем случае к случаю индикатора двоичного куба}
    \subsection{ Лемма о представлении диффеоморфизма в виде композиции диффеоморфизмов специального вида}
    \subsection{ Теорема о замене переменной в кратном интеграле}
    
\end{document}