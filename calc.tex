\documentclass[a4paper]{article}
\usepackage{cmap}
\usepackage[utf8]{inputenc}
\usepackage[T2A]{fontenc}
\usepackage{amsfonts}
\usepackage{amsmath, amsthm}
\usepackage{amssymb}
\usepackage{hyperref}
\usepackage{multicol}
\usepackage{tikz} 

\newcommand\letsymbol{\mathord{\sqsupset}}
\usepackage[russian]{babel}
\renewcommand\qedsymbol{$\blacktriangleright$}
\newtheorem{theorem}{Теорема}[section]
\newtheorem{lemma}{Лемма}[section]
\theoremstyle{definition}
\newtheorem*{example}{Пример}
\newtheorem*{definition}{Определение}
\theoremstyle{remark}
\newtheorem*{remark}{Замечание}
\newtheorem*{corollary}{Следствие}

\setlength{\topmargin}{-0.5in}
\setlength{\oddsidemargin}{-0.5in}
\textwidth 185mm
\textheight 250mm

\begin{document}
    \tableofcontents
    \section{Интегралы, зависящие от параметра}
    \subsection{	Интегралы, зависящие от параметра. Принцип равномерной сходимости}
    $\letsymbol f(x,y): [a, b]\times Y$

    Для $\forall y \in Y f_y(x) = f(x,y) - \letsymbol\quad$ она $\in R([a, b])$ (интегрируема)
    
    $\implies \forall \alpha\quad$ и $\quad\beta\in[a,b]$ определена функция $F(y, \alpha, \beta) = \int_{a}^{b} f_y(x)dx = \int_{a}^{b} f(x, y)dx$

    $F(y, \alpha, \beta)$ - функция, заданная интегралом, зависящим от параметра

    [$F(y, a, b)$ - частный случай функции]

    \begin{definition}
        $X\times Y \subset \mathbb{R}^2, f(x,y)$ определена на $X\times Y$, пусть $y_0$ - предельная точка Y
   \begin{enumerate}
        \item пусть $\forall x \in X \quad \exists \lim\limits_{y\to y_0}f(x,y):=\phi(x)$
        \item пусть $\forall \epsilon >0 \exists \delta(\epsilon)$ такая что $|y - y_0|<\delta |f(x,y) - \phi(x)|< \epsilon$ для $\forall x \implies$ тогда говорят, что $f(x,y)$ равномерно сходится к $\phi(x)$ 
      \end{enumerate}
   \end{definition}

   \begin{theorem}[Свойства равномерной сходимости]
    $f:X \times Y \longrightarrow \mathbb{R}, y_0$ - предельная точка $Y$
\begin{enumerate}
    \item $f(x,y)$ равномерно на $X$ сходится к $\phi(x)$ тогда и только тогда,
     если $\forall \epsilon>0 \quad \exists \delta(\epsilon): \forall x \in X \forall y', y'' \in Y$ $|f(x, y') - f(x, y'')|<\epsilon$ [Критерий Коши]
    \item $f(x,y)$ равномерно по $X$ стремится к $\phi(x)$ тогда и только тогда, если для  $\forall\{y_n\}$ так что $y_n \longrightarrow y_0$
     - последовательность $\{f(x,y_n)\}$ равномерно сходится к $\phi(x)$ [сходимость по Гейне]
    \item Если при $\forall y$ функция $f(x,y)$ непрерывна по x (интегрируема) и $f(x,y)$ равномерно сходится к $\phi(x)$, то $\phi(x)$ -  непрерывна и интегрируема
    \item $\letsymbol x_0, y_0$ предельные точки X и Y, $f(x,y)$ равномерно по х сходится к $\phi(x)$,\hfill \break 
     $\letsymbol\forall y \in Y \exists \lim\limits_{x \to x_0}f(x,y) =: \psi(y)$, тогда $\exists \lim\limits_{x \to x_0}\phi(x) = \lim\limits_{y \to y_0}\psi(y) [= \lim\limits_{x \to x_0}\lim\limits_{y \to y_0}f(x,y)]$ 
  \end{enumerate}
\end{theorem}
\begin{proof}
    \begin{enumerate}
         \item $\triangleleft \Rightarrow  \lim\limits_{y \to y_0}f(x,y) =: \phi(y)$\hfill \break 
         $|f(x,y') - f(x,y'')| = |f(x,y') - \phi(x) - f(x,y'')+\phi(x)| \le |f(x,y') - \phi(x)| + |f(x,y'')-\phi(x)|$\hfill \break
         $\Leftarrow x \in X |f(x,y') - f(x,y'')| < \epsilon$ при
         $
         \begin{array}{l}
              |y_0 - y'|< \delta\\
              |y_0 - y''|<\delta
         \end{array}
         $
         $\Leftarrow$ при $\forall x \exists \lim\limits_{y\to y_0}f(x,y)=:\phi(x)$
    
         $|f(x,y')-f(x,y'')| < \epsilon$, $y''\rightarrow y_0$
    
         $|f(x,y')-\phi(x)| \le\epsilon$ ,$f(x,y) \rightrightarrows \phi(x)$
         \item \hypertarget{p1}{Необходимость очевидна}
         
         Достаточность: $\{y_n\} \rightarrow y_0$
    
         $\{f(x,y_n)\} \rightarrow \phi(x)$, пусть $|y_0 - y_n|< \delta = \frac{1}{n} \implies {\ y_n}\ \rightarrow y_0$ 
         
         и $|f(x,y_n) - \phi(x)|>\epsilon$; $f(x,y_n) \nrightarrow\phi(x)$ противоречие
         \item $\letsymbol \{y_n \} \rightarrow y_0, f_n(x) = f(x, y_n)$
         
         $f_n(x)$ равномерно сходится к $\phi(x)$ по \hyperlink{p1}{2}
    
         Далее $\phi(x)$  равномерный предел хороших функий $\implies \phi(x)$ хорошая

         Попа дробнее... (для последовательности функций от одной переменной)

         $|s(x_0+h) - s(x_0)| = |s(x_0+h) + s_n(x_0 + h) - s_n(x_0) - s_n(x_0+h) + s_n(x_0)- s(x_0)|$ 
         
         $\leq |s(x_0+h) - s_n(x_0 + h)| + |s_n(x_0 + h) - s_n(x_0)| + |s_n(x_0) - s(x_0)|$

         Каждое из этих слагаемых меньше $\epsilon/3$(среднее по причине непрерывности $s_n(x)$, остальные по причине равномерной сходимости)
         \item $f(x,y) \rightrightarrows \phi(x), \letsymbol \epsilon>0$, выберем $\delta >0$ такое что:
         
         $|y_0 - y'| < \delta$ и $|y_0-y''|< \delta \implies$
    
         $|f(x,y') - f(x,y'')|< \epsilon$ по к. Коши
    
         $x \to x_0 : |\psi(y') - \psi(y'')| \leq \epsilon \implies$
         
         для $\psi(y)$ верен критерий Коши $\implies$
    
         $\exists \lim\limits_{y \to y_0}\psi(y) = A = \lim\limits_{y \to y_0}\lim\limits_{x \to x_0}f(x,y)$
    
         $|f(x,y) - \phi(x)|< \epsilon, |\psi(y) - A|< \epsilon$ если $|y - y_0|< \delta$
    
         $|\phi(x)-A| \leq {|\phi(x) - f(x,y)|}_{\leq\epsilon}+{|f(x,y) - \psi(y)|}_{<\epsilon, \text{т.к дельты}} + {|\psi(y) - A|}_{\leq\epsilon} \leq 3\epsilon$
    
         при $x \to x_0 \implies \lim\limits_{x\to x_0}\phi(x) = A$ 
    \end{enumerate}
\end{proof}

    \subsection{	Теорема о коммутировании двух предельных переходов. Предельный переход под  знаком интеграла}

    $f(x,y): [a,b]\times Y\rightarrow\mathbb{R}, y_0$ - предельная точка Y и 
    $f_y(x) = f(x,y)$ - интегрируема на $[a,b]$
    
    $F(y) = \int_{a}^bf(x,y)dx$
    \begin{theorem}[О предельном переходе] \hypertarget{p2}{}
         Если кроме того, что $f(x,y)$ равномерно на $[a,b]$ стремится к $\phi(x)$ при $y\to y_0$, то
         $\lim\limits_{y\to y_0}F(y) = \lim\limits_{y\to y_0}\int_{a}^bf(x,y)dx = \int_{a}^b \lim\limits_{y\to y_0} f(x,y)dx$
    \end{theorem}
    \begin{proof}
         $\triangleleft \phi(x)$ - равномерный предел, непрерывен
    
    $f_y(x)\implies \phi(x) $ - интегрируема, $\letsymbol{} \epsilon > 0 \quad \delta(\epsilon)>0$ выбрано из
    определения равномерной сходимости
    
    $|\int_{a}^bf(x,y)dx - \int_{a}^b\phi(x)dx|$ = $|\int_{a}^b(f(x,y) - \phi(x))dx| \leq \int_{a}^b|f(x,y) - \phi(x)|dx$
    $\leq \epsilon(b-a)$ если $|y - y_0|<\epsilon$
    
    $\lim\limits_{y\to y_0}\int_{a}^b f(x,y)dx = \int_{a}^b \phi(x) dx$
    \end{proof}

    \subsection{	Теорема о непрерывности интеграла, зависящего от параметра}
    \begin{theorem}[Непрерывность]

        $f(x,y) - $непрерывна, $f: [a,b]\times [c,d]\rightarrow \mathbb{R} \implies$
   
        $f(y) = \int_{a}^b f(x,y)dx $ непрерывна на $[c,d]$
   \end{theorem}
   \begin{proof}
        $\triangleleft [a,b]\times [c,d]$ компакт $\implies f(x,y)$ равномерно непрерывна на компакте
   
   $\forall \epsilon>0:$
   $
        \begin{array}{l}
             |x - x'|< \delta\\
             |y - y'|<\delta
        \end{array}
        $
        $\implies |f(x,y) - f(x', y')|<\epsilon$
   
        $x' = x, y' = y_0$
   
        $|f(x,y) - f(x,y_0)|<\epsilon$ при $|y - y_0|< \delta(\epsilon)$
   
        $f(x,y) \rightrightarrows f(x, y_0) = \phi(x)$ равномерный предел не зависит от х 
   
        по теореме о предельном переходе:
        
        $\lim\limits_{y\to y_0} F(y) = \lim\limits_{y\to y_0} \int_{a}^b f(x,y)dx = \int_{a}^b \phi(x) dx = \int_{a}^b f(x,y_0)dx  = F(y_0)\implies$
        $F$ непрерывна в $y_0 \in [c,d]\implies F$ непрерывна на $[c,d] $
   \end{proof}
    \subsection{	Дифференцирование под знаком интеграла. Правило Лейбница}
    \begin{theorem}[О дифференцируемости интеграла, зависящего от параметра]\hypertarget{theorem4}{}
     $f(x,y)$ - определена в $[a, b]\times[c, d]$ при $\forall y \in [c, d]$
     функция $f_y(x) = f(x, y)$ непрерывна по х, $\letsymbol{} f'_y(x,y)\exists$
     и непрерывна в прямоугольнике, тогда

     $F(y) = \int_{a}^{b}f(x,y)dx$ и $F'(y) = \int_{a}^{b}f'_y(x,y)dx$
\end{theorem}
\begin{proof}
$\triangleleft$ в силу непрерывности $f(x,y)$ по х, определена $F(y) = \int_{a}^{b}f(x,y)dx$

$y_0 \in [c,d], F(y_0) = \int_{a}^{b}f(x,y_0)dx$

$F(y_0+\triangle) = \int_{a}^{b}f(x,y_0 + \triangle)dx$

$\frac{F(y_0+\triangle) - F(y_0)}{\triangle} = \int_{a}^{b}\frac{f(x,y_0+\triangle) - f(x,y_0)}{\triangle}dx$

По теореме Лагранжа, $\exists \theta \in (0, 1)$ т.ч 

$\frac{f(x,y_0+\triangle) - f(x, y_0)}{\triangle} = f'_y(x, y_0+\theta\triangle)$

т.к F непрерывна $\implies$ равномерно непрерывна  $\implies$
для $\epsilon>0 \exists \delta>0$
$\begin{array}{l}
|x' - x''|< \delta\\
|y' - y''|<\delta
\end{array}
$
$\implies |f'_y(x', y') - f'_y(x'', y'')|$


$x' = x'' = x, y' = y_0 + \triangle\theta, y'' = y_0,$если $\triangle<\delta$

$|\frac{f(x,y_0+\triangle) - f(x, y_0)}{\triangle} - f_y'(x,y_0)| = |f'_y(x, y_0+\theta\triangle) - f_y'(x,y_0)|<\epsilon$ т.к $\delta(\epsilon)$

неравенство не зависит от точек, т.е

$\frac{f(x,y_0+\triangle) - f(x, y_0)}{\triangle}\rightrightarrows f_y'(x,y_0)$ равномерно по х

В силу теоремы \hyperlink{p2}{о предельном переходе}, получаем что $\int_a^b \frac{f(x,y_0+\triangle) - f(x, y_0)}{\triangle}dx \rightarrow \int_a^b f'_y(x, y_0)dx$

$\frac{F(y_0+\triangle) - F(y_0))}{\triangle} \rightarrow F'_y(y_0)$
\end{proof}
    \subsection{	Интегрирование под знаком интеграла}
    \begin{theorem}[О интегрируемости F(y)] 
     $\letsymbol{}f(x,y)$ непрерывна в $[a, b]x[c, d]$, тогда имеет место равенство

     $\int_c^d (\int_a^b f(x,y)dx)dy = \int_a^b (\int_c^d f(x,y)dy)dx$
\end{theorem}
\begin{proof}
     $\triangleleft \letsymbol{} \eta \in [c, d]$, покажем, что
$\int_c^\eta (\int_a^b f(x,y)dx)dy = \int_a^b (\int_c^\eta f(x,y)dy)dx$

$\int_c^\eta F(y)dy = \mathcal{F}(\eta) -  \mathcal{F}(c), \mathcal{F}' = F$

Производная левой части по $\eta = F(\eta) = \int_a^b f(x,\eta)dx$

$\phi (\eta) := \int_c^\eta f(x,y)dy$ непрерывна по x

$\phi (x, \eta) \rightarrow \phi_\eta'(x, \eta)$

$\letsymbol{} \Phi(x, \eta)'= \phi(x,\eta), \Phi'(x,\eta) = f$

$\phi(x,\eta) = \Phi(x,\eta) - \Phi(x, c)$

$\phi'_\eta = \Phi'_\eta  = f \quad \phi'_\eta(x, \eta) = f(x, \eta)$

По \hyperlink{theorem4}{предыдущей теореме}
$(\int_a^{b}\phi(x, \eta)dx)'_\eta = \int_a^b \phi_\eta'(x, \eta)dx$
$= \int_a^b f(x,\eta)dx = F(\eta)\implies$
левая и правая часть могут отличаться лишь на const, но при $\eta = c$
обе части равны 0 $\implies C = 0$
\end{proof}
    \subsection{	Непрерывность и дифференцируемость интеграла с переменными пределами интегрирования}
    \begin{theorem}
     $\letsymbol{} f(x,y)$ определена и непрерывна в прямоугольнике $[a, b]\times[c, d]$

     $x = \alpha(y); x = \beta(y)$ непрерывны и не выходят за пределы прямоугольника

     Тогда $F(y) = \int_{\alpha(y)}^{\beta(y)}f(x,y)dx$ непрерывен
\end{theorem}

\begin{proof}
     $\triangleleft y_0\in[c,d]$

     $F(y) = \int_{\alpha(y_0)}^{\beta(y_0)}f(x,y)dx + \int_{\beta(y_0)}^{\beta(y)}f(x,y)dx - \int_{\alpha(y_0)}^{\alpha(y)}f(x,y)dx$

     т.к $\beta(y_0), \alpha(y_0) = C$, то 

     $\int_{\alpha(y_0)}^{\beta(y_0)}f(x,y)dx \stackrel{\rm{def}}{=}\widetilde{F}(y)\rightarrow \int_{\alpha(y_0)}^{\beta(y_0)}f(x,y_0)dx = \widetilde{F}(y_0)$

     $|\int_{\beta(y_0)}^{\beta(y)}f(x,y)dx| \leq \int_{\beta(y_0)}^{\beta(y)}|f(x,y)|dx \leq M |\beta(y)-\beta(y_0)|\to 0$, где
     $M \leq |f(x,y)|$, при $y\to y_0$

     при $y \to y_0 \quad F(y) \to \widetilde{F}(y)$

     $F(y)\to \widetilde{F}(y) \to \widetilde{F}(y_0) = F(y_0)$
\end{proof}

\begin{theorem}
     $\letsymbol{}f(x,y)$ определена в $[a,b]\times[c,d]$ имеет в ней непрерывную производную $f'_y(x,y)$

     $\alpha'(y)$ и $\beta'(y)$ - непрерывны, тогда $F'_y(y) = \int_{\alpha(y_0)}^{\beta(y_0)}f'_y(x,y)dx + \beta'(y)f(\beta(y), y) - \alpha'(y)f(\alpha(y), y)$
\end{theorem}

\begin{proof}
     $F(y) = \int_{\alpha(y_0)}^{\beta(y_0)}f(x,y)dx + \int_{\beta(y_0)}^{\beta(y)}f(x,y)dx - \int_{\alpha(y_0)}^{\alpha(y)}f(x,y)dx$

     $(\int_{\alpha(y_0)}^{\beta(y_0)}f(x,y)dx)'_y = \int_{\alpha(y_0)}^{\beta(y_0)}f'_y(x,y)dx$ т.к пределы постоянные

     $\frac{\int_{\beta(y_0)}^{\beta(y)} f(x,y)dx - 0}{y-y_0} = \frac{f(\widetilde{x}, y) (\beta(y) - \beta(y_0))}{y-y_0} [\widetilde{x}$ между $\beta(y)$ и $\beta(y_0)]$

     при $y  \to y_0 \frac{\int_{\beta(y_0)}^{\beta(y)} f(x,y)dx}{y-y_0} \to f(\beta(y_0), y_0)\beta'(y_0)$, т.е
     
     $(\int_{\alpha(y_0)}^{\beta(y_0)}f(x,y)dx)'_y = f(\beta(y), y)\beta'(y)$, аналогично со вторым интегралом
\end{proof}
    \subsection{	Равномерная сходимость интегралов. Достаточные признаки равномерной сходимости}

    $\int_a^\omega F(x)dx$ - несобственный, если $\omega = \pm\infty$ или $f(x)$ не ограничена в окрестности $\omega$

$\letsymbol{} f(x,y)$ определена на множестве $[a, \omega)\times Y$

Для всех $y\in Y$ функция $f_y(x) = f(x,y)$ несобственно интегрируема на $[a, \omega)$, тогда $F(y) = \int_a^\omega f(x,y)dx = \lim\limits_{b\to\omega} \int_a^b f(x,y)$

\begin{definition}
     $f(b, y) = \int_a^b f(x,y) dx$, тогда сходимость F(y) равносильна существованию предела $\lim\limits_{b\to\omega}F(b, y) = F(y) = F(\omega, y)$
\end{definition}
\begin{definition}
     F(y) называется равномерно сходящейся относительно y на Y, если $\forall\epsilon \quad\exists\delta(\epsilon): \forall y\in Y \quad \forall b \in (a,\omega) |b-\omega|< \delta \implies |F(b, y) - F(y)| < \epsilon$

     $F(b, y)\rightrightarrows_{b\to \omega} F(y)$
\end{definition}
\begin{remark}
     $\letsymbol{} - \{b_n\}$ - последовательность сходится к $\omega$ согласно свойствам равномерной сходимости

     $F(b,y) \rightrightarrows F(y) \leftrightarrow F(b_n, y) \rightrightarrows F(y)$

     $a_n y \stackrel{\rm{def}}{=} \int_{b_n}^{b_{n+1}}f(x,y)dx, b_1 = a, b_j \geq a$

     Тогда $F(y) = \sum_{n = 1}^{\infty} a_n(y) $

     Равномерная сходимость F(y) равносильна равномерной сходимости ряда
\end{remark}

\begin{theorem}[Признаки равномерной сходимости интеграла]
     \begin{enumerate}
          \item(Вейерштрасса) $f(x,y)$ определена на $[a, \omega)\times Y, \omega$ - особая точка f(x,y) и f(x,y) интегрируема на $[a,b]\subset[a, \omega)$
          Если $\exists \phi(x) такая что |f(x,y)| \leq\phi(x)\quad \forall x \in [a,\omega)\forall y\in Y$ и $\int_a^\omega \phi(x)dx$ сходится, то $\int_a^\omega f(x, y)dx = F(y)$
          \item(Дирихле) $F(y) = \int_a^\omega f(x, y)g(x,y)dx, g(x,y)$ монотонно по х -> $\omega$ равномерно по y стремится к 0
          и для $\forall$ отрезка $[a,b]\subset[a,\omega)$

          $|\int_a^b f(x, y)dx|\leq L$, тогда F(y) сходится равномерно
          \item (Абель) $F(y) = \int_a^\omega f(x, y)g(x,y)dx$
          
          Если $\int_a^\omega f(x, y)dx$ сходится равномерно $g(x,y)$ монотонно по х равномерно по у сходится к своему пределу 
     \end{enumerate}
\end{theorem}
\begin{proof}
     \begin{enumerate}
          \item очевидно
          Для F(y) используем критерий Коши
          \item $\int_{b'}^{b''}f(x,y)g(x,y)dx = g(b', y)\int_{b'}^\xi f(x,y)dx + g(b'', y)\int_{\xi}^{b''} f(x,y)dx, \xi \in (b', b'')$
          
          $g(b, y)\to 0$ равномерно по y $\implies\exists B$ такое что $\forall b', b'' > B$

          $|g(b', y)|< \frac{\epsilon}{2L}\quad |g(b'', y)|< \frac{\epsilon}{2L}\implies F(y)$ сходится равномерно
          \item $\int_a^\omega f(x,y)dx$ сходится равномерно
          $\forall \epsilon>0 \exists\delta \quad \forall b', b'' > B |\int_{b'}^{b''}f(x,y)dx| \widetilde{\epsilon}$

          т.к $g(x,y)$ равномерно сходится к G(y)

          $|g(x,y)|\leq M$ при х близком к $\omega$

          $\widetilde{\epsilon} = \frac{\epsilon}{2M}$, $|\int_{b'}^{b''}f(x,y)g(x,y)dx|\leq M\frac{\epsilon}{2M} +M\frac{\epsilon}{2M} = \epsilon\implies F(y)$ сходится равномерно
     \end{enumerate}
\end{proof}
    \subsection{	Предельный переход в несобственном интеграле, зависящем от параметра}
    \begin{theorem}[О предельном переходе] \hypertarget{p4}{}
     $\letsymbol{}f(x,y)$ определена на $[a,\omega)\times Y$ для $\forall y\in Y$, интегрируема на $[a,b]\subset[a, \omega]$ равномерно относительно у сходится к функции $\phi(x)$ при $y\to y_0$ если $F(y) = \int_a^\omega f(x,y)dx$ 
     сходится равномерно относительно $y\in Y$
     $\lim_{y \to y_0}  \int_a^\omega  f(x,y)dx = \int_a^\omega  \phi(x)dx = \int_a^\omega  \lim_{y \to y_0}  f(x,y)dx$ 
\end{theorem}

\begin{proof}
     $F(b, y) = \int_a^b f(x,y)dx$ это несобственный интеграл и для него верна теорема о \hyperlink{p2}{о предельном переходе}

     $\lim_{y \to y_0}F(b, y) = \int_{a}^{b} \phi(x) \,dx $

     $\lim_{b\to \omega}F(b, y) = \int_a^\omega f(x,y)dx$ - равномерно

     $F(b,y)$ - для этой функции верны условии о перемене предельных переходов $\implies$

     $\lim_{y \to y_0} \lim_{b \to \omega} \int_a^b f(x,y)dx = lim_{y \to y_0} \int_a^\omega f(x,y)dx$
\end{proof}
Следствие:
Если $f(x,y)$ монотонно по y $\lim_{y \to y_0}  f(x,y) = \phi(x)$ - непрерывны, тогда

$\int_a^\omega \phi(x)dx\rightrightarrows\int_a^\omega f(x,y)dx$ сходится равномерно

$\lim_{y \to y_0} F(y) = \int_a^\omega \phi(x)dx$
\begin{proof}
     $f(x,y)\to \phi(x)\quad y\to y_0 \quad \forall\epsilon>0\exists\delta: |y - y_0|< \delta\implies |f(x,y) - \phi(x)|< \epsilon$

     $\letsymbol{} f(x,y)$ возрастает по y, тогда $F(b, y) = \int_a^b f(x,y)dx$ возрастает по у

     но $f(x,y)\leq\phi(x)\implies F(b, y)\leq \int_a^b \phi(x)dx \leq \int_a^\omega \phi(x)dx\implies \lim_{b \to \omega} F(b,y) = \int_a^\omega f(x,y)dy$ - сходится

     Равномерность по Вейерштрассу
\end{proof}
    \subsection{	Дифференцирование  по параметру несобственного интеграла}
    \subsection{	Интегрирование по параметру несобственного интеграла}
    
    \section{Кратные интегралы}
    \subsection{ Двоичные разбиения. Двоичные интервалы, полуинтревалы, кубы. Свойства двоичных инервалов, кубов}
    \subsection{ Ступенчатые функции. Интеграл от ступенчатой функции (естественное и индуктивное определения). Теорема о совпадении определений}
    \subsection{ Свойства интеграла от ступенчатой функции (линейность интеграла, положительность, оценка интеграла)}
    \subsection{ Теорема о пределе интегралов убывающей последовательности функций, поточечно сходящейся к нулю}
    \subsection{ Теорема о пределе интегралов убывающей последовательности ступенчатых функций, поточечно сходящейся к нулю}
    \subsection{ Системы с интегрированием. Основной пример. Свойства систем с интегрирование}
    \subsection{ L1 норма. Множество L1*($\Sigma$). L1-норма как интеграл от модуля функции}
    \subsection{ Свойства L1 нормы ("линейность", норма функции равной нулю почти всюду и т.д.)}
    \subsection{ Субаддитивность L1-нормы}
    \subsection{ Сходимость в смысле L1}
    \subsection{ Определение понятие интеграла и интегрируемой функции}
    \subsection{ Свойства интеграла и интегрируемых функций}
    \subsection{ Множества меры ноль. Свойства функций совпадающих почти всюду}
    \subsection{ Нормально сходящиеся ряды. Теорема о нормально сходящихся рядах}
    \subsection{ Теоремы Леви для функциональных рядов и последовательностей}
    \subsection{ Огибающие для последовательности интегрируемых функций. Нижний и верхний предел последовательности}
    \subsection{ Теорема Фату о предельном переходе. Следствие из теоермы Фату}
    \subsection{ Теорема Лебега о предельном переходе}
    \subsection{ Лемма о приближении стпенчатой функции с помощью непрерывных финитных}
    \subsection{ Теорема о приближении интегрируемой функции с помощью непрерывных финитных}
    \subsection{ Измеримые функции. Свойства пространства измеримых функций. Измеримые множества}
    \subsection{ Теорема об интегрируемости измеримой функции}
    \subsection{ Теорема об измеримости предела измеримых функций}
    \subsection{ Теорема об интегрируемости предела возрастающей
    последовательности положительных измеримых функций}
    \subsection{ Обобщенно измеримые функции. Измеримые множества, мера множества. Теорема об измеримости объединения и пересечения измеримых множеств}
    \subsection{ Счетная аддитивность интеграла и меры}
    \subsection{ Измеримые множества в Rn. Внешняя мера множества. Лемма о представлении открытого множества как объединения кубов. Теорема об измеримости открытых и замкнутых множеств в Rn}
    \subsection{ Теорема о внешней мере множества}
    \subsection{ Лемма о приближении неотрицательной вещественной функции ступенчатыми функциями. Следствие об измеримости непрерывной почти всюду функции}
    \subsection{ Теорема о совпадении интералов Римана и Лебега}
    \subsection{ Теорема Фубини и следствия из нее}
    \subsection{ Теорема Тонелли и следствия из нее}
    \subsection{ Диффеоморфизмы и их свойства. Теорема о замене переменной в кратном интеграле (формулировка)}
    \subsection{ Лемма о замене переменной при композиции диффеоморфизмов}
    \subsection{ Лемма о сведении замены переменной в общем случае к случаю индикатора двоичного куба}
    \subsection{ Лемма о представлении диффеоморфизма в виде композиции диффеоморфизмов специального вида}
    \subsection{ Теорема о замене переменной в кратном интеграле}
    
\end{document}