\documentclass[a4paper, 12pt]{article}
\usepackage[utf8]{inputenc}
\usepackage[T2A]{fontenc}
\usepackage{amsfonts}
\usepackage{amsmath, amsthm}
\usepackage{amssymb}
\usepackage{hyperref}
% \usepackage{mathrm}
\usepackage{bbold}
\usepackage{mathrsfs}


\usepackage[russian]{babel}
\renewcommand\qedsymbol{$\blacktriangleright$}
\newtheorem{theorem}{Теорема}[section]
\newtheorem{lemma}{Лемма}
\theoremstyle{definition}
\newtheorem{example}{Пример}
\newtheorem{problem}{Упражнение}[section]
\newtheorem*{definition}{Определение}
\theoremstyle{remark}
\newtheorem*{remark}{Замечание}
\newcommand\letsymbol{\mathord{\sqsupset}}


\DeclareMathOperator{\degg}{deg}

\setlength{\topmargin}{-0.5in}
\setlength{\oddsidemargin}{-0.5in}
\textwidth 185mm
\textheight 250mm

\begin{document}
\section*{3-раскраска графа}
\textbf{Вход:} неориентированный граф без кратных ребер и петель $G = (V, E)$ c n вершинами.
\\
\textbf{Выход:} $\begin{cases}
    1,\text{если существует правильная 3-раскраска графа,}\\
    0,\text{ иначе.}
\end{cases}$
\begin{lemma}[Геллер]
    $G(V,E)$ -- произвольный связный неориентированный граф с $n$ вершинами и $m$ ребрами. Тогда $\dfrac{n^2}{n^2 - 2m} \leqslant \chi(G) $
\end{lemma}
\begin{proof}
    Пусть, $V_1,V_2...V_\chi$ - множества вершин, окрашенных в соответствующие цвета при правильной покраске графа $G$.

    $$m \leqslant \dfrac{1}{2}n(n - 1) - \dfrac{1}{2}\sum\limits^{\chi}_{i = 1}|V_i|(|V_i| - 1) \Rightarrow \dfrac{n^2}{n^2 - 2m} \leqslant \dfrac{n^2}{n^2 -n(n - 1) + \sum\limits^{\chi}_{i = 1}|V_i|(|V_i| - 1)} =$$

$$=  \dfrac{n^2}{n + \sum\limits^{\chi}_{i = 1}|V_i|(|V_i| - 1)} =\dfrac{n^2}{\sum\limits^{\chi}_{i = 1}|V_i| + \sum\limits^{\chi}_{i = 1}|V_i|(|V_i| - 1)} = \dfrac{n^2}{\sum\limits^{\chi}_{i = 1}|V_i|^2} = \dfrac{(\sum\limits^{\chi}_{i = 1}|V_i|)^2}{\sum\limits^{\chi}_{i = 1}|V_i|^2} \leqslant \chi.$$
\end{proof}
Для обеспечения невозможности 3-раскраски необходимо, чтобы $\chi(G) \geqslant 4$. Тогда
\[\frac{n^2}{n^2-2m} \geqslant 4\implies 8m \geqslant 3n^2\]
\subsection*{Генерический алгоритм}
\begin{enumerate}
    \item Если $8|E| \geqslant 3{|V|}^2$, то правильной 3-раскраски не существует.

    Иначе ответ - "не знаю".
    % \item Сравнить $|E|$ и $\frac32|V|=\frac32 n$.

    % Если $|E| \ge \frac32 n$, то правильной 3-раскраски не существует.
    
    % Иначе ответ - "не знаю"
\end{enumerate}
\begin{proof}
    \[I_n = \{G = (V, E) \;| \;|V| = n\}, |I_n| = 2^{\frac{(n-1)n}{2}}\]
    \[S = \{G = (V, E) \;| \; 8|E| < 3{|V|}^2\}\]
    \[S \cap I_n = \{G = (V, E) \;| \;|V| = n, |E| = m, 8m < 3{n}^2\}\]
    Рассмотрим матрицы смежности неориентированных (m, n)-графов. Единиц в таких матрицах в два раза больше, чем ребер в графе, поэтому верно следующее:
    \[|S \cap I_n| = |\{\text{множество симметричных булевых матриц n x n с нулями на главной диагонали, }\]
    \[\text{у которых выше главной диагонали единиц меньше, чем }\frac38 n^2\}| = \]

\end{proof}
    $|I_n \setminus (I_n\cap S)|$ - множество графов такое, что $\forall v\in V \degg(v) \ge 3$, т.е для них не существует правильной 3-раскраски
    (каждой вершине инцидентно не менее трех других вершин, а у нас лишь 3 цвета для раскраски, следовательно, по принципу Дирихле,
    как минимум у 2 вершин совпадают цвета).

    Для множества S: $2|E| = \sum_{v\in V} \degg (v) < 3|V|$.
    \[S = \{G = (V, E) \;|\; |E| <\frac32 |V|\} = \{M \;|\; \text{число единиц в матрице} < 3|V|\}\]

    \[S_1 = \{G = (V, E) \;|\; \text{графы с множеством матриц (смежности) такие, что в каждой строчке}\]
    \[\text{единиц правее главной диагонали меньше или равно 2}\}\]
    Заметим, что $S_1 \supseteq S$, т.к. для матриц из $S_1$ верно следующее - если правее главной диагонали в каждой строке единиц меньше или равно 2, то в каждой строке единиц меньше или равно 4, т.е всего единиц в матрице $\le 4 |V| = 4n$.

    Мы рассматриваем симметричные матрицы смежности (т.к неориентированный граф). Тем самым:
    
    \[|I_n| = 2^{\frac{(n-1)n}{2}}\]
    \[|I_n\cap S_1| = 2\prod_{i=1}^{n-2} ((n-i)(n-i-1) + n - i +1)\]

    \[ \rho (S_1) = \lim_{n\to \infty} \frac{2\prod_{i=1}^{n-2}((n-i)(n-i-1) + n - i +1)}{2^{\frac{(n-1)n}{2}}}\]

    Заметим, что сверху полином $n^{2(n-2)}$ степени, поэтому предел равен 0.

    (т.к $\lim_{n\to \infty}\frac{n}{2^n} = 0$)

    Подмножество пренебрежимого множества пренебрежимо, тем самым $S$ также пренебрежимо. Алгоритм является генерическим.

    v
\end{document}