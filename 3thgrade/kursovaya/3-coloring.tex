\documentclass[a4paper, 12pt]{article}
\usepackage[utf8]{inputenc}
\usepackage[T2A]{fontenc}
\usepackage{amsfonts}
\usepackage{amsmath, amsthm}
\usepackage{amssymb}
\usepackage{hyperref}
% \usepackage{mathrm}
\usepackage{bbold}
\usepackage{mathrsfs}


\usepackage[russian]{babel}
\renewcommand\qedsymbol{$\blacktriangleright$}
\newtheorem*{theorem}{Утверждение}
\newtheorem{lemma}{Лемма}
\theoremstyle{definition}
\newtheorem{example}{Пример}
\newtheorem{problem}{Упражнение}[section]
\newtheorem*{definition}{Определение}
\theoremstyle{remark}
\newtheorem*{remark}{Замечание}
\newcommand\letsymbol{\mathord{\sqsupset}}


\DeclareMathOperator{\degg}{deg}

\setlength{\topmargin}{-0.5in}
\setlength{\oddsidemargin}{-0.5in}
\textwidth 185mm
\textheight 250mm

\begin{document}
\section*{3-раскраска графа}
\textbf{Вход:} неориентированный граф без кратных ребер и петель $G = (V, E)$ c n вершинами.
\\
\textbf{Выход:} $\begin{cases}
    1,\text{если существует правильная 3-раскраска графа,}\\
    0,\text{ иначе.}
\end{cases}$
\subsection*{Предлагаемый алгоритм} \label{algo}
Если граф является надграфом $K_4$, то возвращаем 0. \\
Иначе "?".
\begin{theorem}
    Алгоритм \ref{algo} является генерическим.
\end{theorem}
\begin{proof}
    Граф $K_4$ нельзя раскрасить в 3 цвета, значит, если он является подграфом некоторого графа, то сам
    граф также нельзя раскрасить тремя цветами.
    
    Докажем, что множество графов, не содержащих подграфа $K_4$ (обозначим его $G_n$), является пренебрежимым.
    
    Граф определяется своей матрицей смежности. Количество матриц смежности графов размера n - $|I_n| = 2^{\frac{n(n-1)}{2}}$
    
    $$|G_n| = |\{\text{графы размера n, матрицы смежности которых }$$
    $$ \text{не содержат подматрицы вида }J_{4\times 4} - E_{4}\}|$$

    В частности, на диагонали расположены $n/4$ подматрицы $4\times 4$, разрешенных вариантов по $2^6 - 1$.

    Если мы рассмотрим множество матриц, где запрет наложен только на $n/4$ диагональных подматриц (обозначим множество $G_n'$), а не на все $C^4_n$, то получим большее по мощности множество.

    $$|G_n'| = (2^6-1)^{n/4} \cdot 2^{\frac{n(n-1)}{2} - \frac{6n}{4}} = (2^6-1)^{n/4} \cdot 2^{\frac{n^2 - 4n}{2}}$$

    $$\mu(G') = \lim_{n\to \infty} \frac{|G_n'|}{|I_n|} = \lim_{n\to \infty} \frac{(2^6-1)^{n/4} \cdot 2^{\frac{n^2 - 4n}{2}}}{2^{n(n-1)\over 2}} = \lim_{n\to\infty} {(2^6-1)^{n/4}}\cdot 2^{-\frac{3}{2}n} = \lim_{n\to \infty} (\frac{2^6 - 1}{2^6})^{n/4} = 0$$

    Получаем, что множество $G'$ пренебрежимо. Очевидно, что $G_n$ также пренебрежимо, ибо $G_n\subseteq G_n'$.

    
\end{proof}
\end{document}