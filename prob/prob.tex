\documentclass[a4paper]{article}
\usepackage{cmap}
\usepackage[utf8]{inputenc}
\usepackage[T2A]{fontenc}
\usepackage{amsfonts}
\usepackage{amsmath, amsthm}
\usepackage{amssymb}
\usepackage{hyperref}
\usepackage{multicol}
\usepackage{xcolor}

\newcommand\letsymbol{\mathord{\sqsupset}}
\usepackage[russian]{babel}
\renewcommand\qedsymbol{$\blacktriangleright$}
\newtheorem{theorem}{Теорема}[section]
\newtheorem{lemma}{Лемма}[section]
\newtheorem*{axiom}{Аксиома}
\theoremstyle{definition}
\newtheorem*{definition}{Определение}
\newtheorem*{statement}{Утверждение}
\theoremstyle{remark}
\newtheorem*{remark}{Замечание}

\setlength{\topmargin}{-0.5in}
\setlength{\oddsidemargin}{-0.5in}
\textwidth 185mm
\textheight 250mm

\begin{document}
\tableofcontents
\section{Вероятностное пространство}

\begin{definition}[Алгебра]
    Семейство $\mathcal{A}$ подмножеств множества $\Omega$ называется
    алгеброй, если выполнены след. аксиомы:
    \begin{enumerate}
        \item $\varnothing \in \mathcal{A}$
        \item $A\in\mathcal{A} \implies \overline{A}\in \mathbb{A}$
        \item (аддитивность) $A_1, \dots, A_n \in \mathbb{A} \implies A_1 \cup \dots \cup A_n\in \mathbb{A}$ 
    \end{enumerate}
\end{definition}
\begin{definition}[$\sigma$-алгебра]
    Алгебра называется $\sigma$-алгеброй, если 
    \[A_1, \dots, A_n \in \mathcal{A} \implies \bigcup\limits_{k = 1}^\infty  A_k \in \mathcal{A}\]
\end{definition}
\begin{definition}[мера]
    $\mu: \mathcal {A} \to [0; \infty)$ - мера, если 
    \[A_1, ..., A_n \in \mathcal{A}, A_i\cap A_j = \varnothing, i\neq j: \quad \mu(\bigcup\limits_{n = 1}^\infty  A_n) = \sum\limits_{n = 1}^\infty \mu (A_n)\quad \text{счетная аддитивность}\]

    Мера конечная, если $\mu(\Omega) < \infty$

    Мера вероятностная, если $\mu (\Omega) = 1$
\end{definition}
\begin{definition}[Вероятностное пространство]
    Тройка $(\Omega, \mathcal{A}, P)$, где 
    \begin{enumerate}
        \item $\Omega$ - пространство элементарных событий;
        \item $\mathcal{A}$ - $\sigma-$алгебра подмножеств $\Omega$ (события);
        \item P - вероятностная счетно-аддитивная мера на $\mathcal{A}$ (вероятность);
        называется вероятностным пространством.
    \end{enumerate}
\end{definition}
Все элементарные исходы равновозможны
\begin{multicols*}{2}
    \begin{definition}[Классическая вероятность]
        Модель вероятностного пространства (A - событие)
        \begin{enumerate}
            \item $\Omega = \{\omega_1, \dots, \omega_n\}$ - конечное пространство
            \item $\mathcal{A}$ все подмножества $\Omega$
            \item $P(A) = \sum\limits_{\omega\in A}p_\omega = \frac{|A|}{|\Omega|}$
        \end{enumerate}
    \end{definition}
    \vfill\null\columnbreak
    \begin{definition}[Геометрическая вероятность]  
        $V\in \mathbb{R}^n$
        \begin{enumerate}
            \item $\Omega = V$
            \item $\mathcal{A}$ - борелевская $\sigma-$алгебра 
            (минимальная $\sigma-$алгебра, содержащая все компакты) подмножеств $V$
            \item $P(A) = \frac{\mu(A)}{\mu(V)}$
        \end{enumerate}
    \end{definition}
\end{multicols*}

\subsection{Некоторые следствия аксиоматики}
\begin{enumerate}
    \item \begin{axiom}[Аксиома непрерывности]
        Если $A_1\supset A_2, \dots, \supset A_n \supset \mathcal{A}, \bigcap\limits_{i = 1}^\infty A_i = \varnothing$, то
        \[\lim\limits_{n\to \infty} P(A_n) = 0\]
    \end{axiom}
    \begin{proof}
        Пусть $B_n\downarrow \varnothing$. Тогда обозначим $A_n = B_n \setminus B_{n+1}, n = 1, \dots, .$. $A_n$ попарно несовместны и 
        \[B_1  =    \sum\limits_{n = 1}^\infty A_n \quad B_n  =  \sum\limits_{k = n}^\infty A_k, \]
        поэтому из счетной аддитивности меры следует сходимость ряда \[P(B_1)=\sum\limits_{n = 1}^\infty P(A_n),\] и сумма остатка ряда \[P(B_n) =  \sum\limits_{k = n}^\infty P(A_k) = 0.\] 
    \end{proof}
    \item (Формула включений и исключений)
    \[P(\bigcup\limits_{k = 1}^\infty A_k)  = \sum_{k = 1}^n P(A_k) - \sum_{i < j}^n P(A_i \cap A_j) + \dots + {(-1)}^{n-1} P(A_1 \cap ... \cap A_n)\]
    \begin{proof}
        Выводится через обычную формулу включений и исключений для множеств по индукции \[P(A \cup B) = P(A) + P(B) - P(AB)\]+ 
        \[\begin{cases}
            A \cup B = A + (B\setminus AB) \\ 
            \text{Счетная аддитивность} \\ 
            P(B \setminus AB) = P(B) - P(AB) (\text{также по счетной аддитивности})
        \end{cases}
        \]
    \end{proof}
\end{enumerate}
\subsubsection{Индикатор}
\begin{definition}
    Индикатор события А - это функция $I_A(\omega) = \begin{cases}
        1,  & \omega \in A \\
        0,  & \omega \notin A
        \end{cases}$
\end{definition}
\paragraph{Свойства индикатора}
\begin{enumerate}
    \item $I_{\bar{A}} = 1 - I_A$
    \item $I_{A_1 \cap A_2} = I_{A_1}I_{A_2}$
    \item $I_{A_1\cup\dots\cup A_n} = 1 - I_{\bar{A_1} \cap \dots\cap \bar{A_n}} = 1  -  I_{\bar{A_1}}\dots I_{\bar{A_n}} = 1 - (1 - I_{A_1})\dots(1 - I_{A_n})$
\end{enumerate}
\section{Условные вероятности и независимость}
\section{Случайные величины}
\begin{definition}[Случайная величина]
    Случайной величиной (СВ) $X(\omega)$ называется функция элементарного события $\omega$ с областью определения $\Omega$
и областью значений $\mathbb{R}$ такая, что событие $\{\omega : X(\omega) \leq x\}$ принадлежит $\sigma$ -алгебре $\mathcal{F}$ при любом действительном $x \in \mathbb{R}$ . Значения x функции $X(\omega)$ называются реализациями СВ $X(\omega)$.    
\end{definition}
\begin{definition}[Закон распределения]
    Любое правило (таблица, функция), позволяющее находить вероятности всех возможных событий, связанных со случайной величиной.
\end{definition}
\paragraph*{Примеры законов распределения}
\begin{definition}[Математическое ожидание]
    Математическое ожидание случайной величины $\xi = xi(\omega)$ обозначается $M\xi$ и определяется как сумма
    \[M\xi = \sum_{\omega \in \Omega} \xi(\omega) p(\omega)\]
\end{definition}
\paragraph*{Свойства мат. ожидания}
\begin{enumerate}
    \item $M I_A = P(A)$
    \begin{proof}
    \[M I_A = \sum_{\omega \in \Omega} I_A(\omega) p (\omega) = \sum_{\omega \in A} p(\omega)  = P(A)\]
    \end{proof}
    \item Аддитивность: $M(\xi + \eta) = M\xi + M\eta$
    \begin{proof}
        
    \end{proof}
    Из этого также следует конечная аддитивность.
    \item Для любой константы C \[M(C\xi) = cM\xi,\quad MC = C\]
    \item Математическое ожидание $\xi$ выражается через закон распределения случайной величины $\xi$ формулой
    \[M\xi = \sum_{i = 1}^k x_k P \{\xi = x_i\}\]
\end{enumerate}
Подставляя в числовую функцию случайную величину, мы также получаем случайную величину. Например, если $\eta = g(\xi)$, то \[M\eta  = M g(\xi)  = \sum_{i = 1}^k g(x_i) P\{\xi = x_i\}\]
При этом \[g(x_i) = \sum_{i = 1}^k g(x_i) I_{\xi = x_i}\]
\end{document}