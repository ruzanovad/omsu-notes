\documentclass[a4paper]{article}
\usepackage{cmap}
\usepackage[utf8]{inputenc}
\usepackage[T2A]{fontenc}
\usepackage{amsfonts}

\usepackage{amsmath, amsthm}
\usepackage{amssymb}
\usepackage{mathrsfs}
\usepackage{hyperref}
\usepackage{multicol}
\usepackage{xcolor}

\newcommand\letsymbol{\mathord{\sqsupset}}
\usepackage[russian]{babel}
\renewcommand\qedsymbol{$\blacktriangleright$}
\newtheorem{theorem}{Теорема}[section]
\newtheorem{lemma}{Лемма}[section]
\newtheorem*{axiom}{Аксиома}
\theoremstyle{definition}
\newtheorem*{definition}{Определение}
\newtheorem*{statement}{Утверждение}
\theoremstyle{remark}
\newtheorem*{remark}{Замечание}

\setlength{\topmargin}{-0.5in}
\setlength{\oddsidemargin}{-0.5in}
\textwidth 185mm
\textheight 250mm

\begin{document}
% \begin{center}
%     \begin{tabular}{|c|c c c c c|}
%         \hline
%         $ $ & $P\{\xi=k\}$ & $M(\xi)$ & $D(\xi)$ & $\phi(s)$ & $f_xi(t)$\\ 
%         \hline
%         Равномерное дискретное & $\frac{1}{N}$ & $\frac{1+N}{2}$ & $\frac{N^2-1}{12}$ & $\sum_{n=1}^{\infty} \frac{s^n}{n}=-\ln(1-s)$ & $ $ \\
%         \hline
%         Биномиальное (распределение Бернулли) & $C_n^k p^k {(1-p)}^{n-k}$ & $np$ & $np(1-p)$ & $\sum_{m = 0}^{\infty} C_n^m p^m {(1-p)}^{n-m} = {(ps +1-p)}^n$ & $ $ \\
%         \hline
%         Геометрическое & $(1-p)p^k$ & $\frac{p}{1-p}$ & $\frac{p}{{(1-p)}^2}$ & $\sum_{n=1}^{\infty}p^k (1-p) s^n =\frac{p}{1-(1-p)s}$ & $ $ \\
%         \hline
%         Пуассоновское & $\frac{\lambda^k}{k!}e^{-\lambda}$ & $\lambda$ & $\lambda$ & $\sum_{n = 0}^{\infty} \frac{\lambda^n s^n}{n!}e^{-\lambda}=e^{\lambda (s-1)}$ & $ $ \\ 
%         \hline
%     \end{tabular}
% \end{center}
\begin{enumerate}
    \item Равномерное дискретное распределение
    \[P\{\xi=k\} = \frac{1}{N}, \quad M\xi = \frac{1+N}{2}, \quad D\xi = \frac{N^2-1}{12}, \quad \phi(s) = \sum_{n=1}^{\infty} \frac{s^n}{n}=-\ln(1-s), \quad f_\xi (t) = -\ln (1-e^{it})\]
    \item Биномиальное (распределение Бернулли)
    \[P\{n=k\}=C_n^k p^k {(1-p)}^{n-k}, \quad M\xi = np, \quad D\xi = np(1-p), \quad \phi(s) = \sum_{m = 0}^{\infty} C_n^m p^m {(1-p)}^{n-m} = {(ps +1-p)}^n,\]
    \[f_\xi (t) = {(pe^{it} +1-p)}^n\]
    \item Геометрическое распределение
    \[P\{n=k\}=(1-p)p^k, \quad M\xi = \frac{p}{1-p}, \quad D\xi = \frac{p}{{(1-p)}^2}, \quad \phi(s) = \sum_{n=1}^{\infty}p^k (1-p) s^n =\frac{1-p}{1-ps}, \quad f_\xi (t) = \frac{1-p}{1-pe^{it}}\]
    \item Распределение Пуассона
    \[P\{n=k\}=\frac{\lambda^k}{k!}e^{-\lambda}, \quad M\xi = \lambda, \quad D\xi = \lambda, \quad \phi(s) = \sum_{n = 0}^{\infty} \frac{\lambda^n s^n}{n!}e^{-\lambda}=e^{\lambda (s-1)}, \quad f_\xi (t) = e^{\lambda (e^{it}-1)}\]
\end{enumerate} 
\begin{remark}[Вырожденное распределение]
    \[P\{\xi = C\} = 1, \quad f_\xi (t) = e^{itC}\]
\end{remark}
\begin{center}
    \begin{tabular}{ | c |c c c c c| }
        \hline
        $ $ & $p_\xi(x)$ & $F_\xi(x)$ & $M(\xi)$ & $D(\xi)$ & $f_\xi(t)$\\ 
        \hline
        гауссовское(нормальное) & $\frac1{\sqrt{2\pi}\sigma} e^{-\frac{(x-a)^2}{2\sigma^2}}$ & $\frac12 [1 + erf(\frac{x-a}{\sqrt{2\sigma^2}})]$ & a  & $\sigma^2$ & $\exp (ita - \sigma^2 t^2/2)$\\ 
        
        равномерное & ${\displaystyle\left\{{\begin{matrix}{\dfrac {1}{b-a}},&x\in [a,b]\\0,&x\not \in [a,b]\end{matrix}}\right..}$ &
        ${\displaystyle \left\{{\begin{matrix}0,&x<a\\{\dfrac {x-a}{b-a}},&a\leqslant x<b\\1,&x\geqslant b\end{matrix}}\right..}$ &
        $\frac{a+b}2$ & $\frac{{(b-a})^2}{12}$ & $ \frac{e^{ita} - e^{itb}}{it (b-a)}$ \\


        гамма распределение & $\displaystyle\left\{{\begin{matrix}x^{{\alpha-1}}{\frac  {e^{{-x\lambda }}}{\lambda ^{-\alpha}\,\Gamma (\alpha)}},&x\geq 0\\0,&x<0\end{matrix}}\right.$ & $\dots$ & $\alpha \lambda^{-1}$ & $\alpha \lambda^{-2}$ & $ (1-it)^{-\alpha} \; \forall \alpha \in \mathbb{Q}$\\
     
        показательное распределение со сдвигом & 
        $\lambda e^{-\lambda (x-b)}, x\ge b$ & $1 - e^{-\lambda (x-b)}$
        & 
        $\frac1{\lambda}$&$\frac{1}{\lambda^2}$
        &
        $\left(1 - \frac{it}{\lambda}\right)^{-1}\text{ для с.в. без сдвига}$\\
     \hline
    \end{tabular}
    \end{center}
    \[\operatorname {erf}\,x={\frac  {2}{{\sqrt  {\pi }}}}\int \limits _{0}^{x}e^{{-t^{2}}}\,{\mathrm  d}t.\]
Выборочное среднее

\[\bar x = \frac1m \sum_{i=1}^m x_i.\]
Выборочная дисперсия:
\[s^2 = s_m^2 = \frac1m \sum_{i=1}^m \left( x_i - \bar x \right)^2\]
Несмещённая оценка дисперсии:
\[s^2 = s_m^2 = \frac1{m-1} \sum_{i=1}^m \left( x_i - \bar x \right)^2.\]
Выборочный момент $k$-го порядка (выборочное среднее — момент первого порядка):
\[M_k = \frac1m \sum_{i=1}^m x^k_i\]
Выборочный центральный момент $k$-го порядка (выборочная дисперсия — центральный момент второго порядка):
\[\overset{\circ}M_k = \frac1m \sum_{i=1}^m \left( x_i - \bar x \right)^k\]
Несмещённые оценки центральных моментов:
\[\overset{\bullet}M_2 = \frac{m}{m-1} \overset{\circ}M_2\]
\[\overset{\bullet}M_3 = \frac{m^2}{(m-1)(m-2)} \overset{\circ}M_3\]
\[\overset{\bullet}M_4 = \frac{m(m^2-2m+3)\overset{\circ}M_4 + 3m(2m-3)\overset{\circ}M_2^2}{(m-1)(m-2)(m-3)}\].
Эмпирическая функция распределения:
\[F_n(t) = \frac{1}{n} \sum_{i=1}^n \mathbf{1}_{X_i \le t}\]
Младшая

 порядковая статистика, функция распределения:
\[F_{X_{(1)}}(x) = 1 - {(1 - F(x))}^n\]
Старшая порядковая статистика, функция распределения:
\[F_{X_{(n)}}(x) = F^n(x)\]
\end{document}
