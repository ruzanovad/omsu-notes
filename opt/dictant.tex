\documentclass[a4paper]{article}
\usepackage{cmap}
\usepackage[utf8]{inputenc}
\usepackage[T2A]{fontenc}
\usepackage{amsfonts}
\usepackage{amsmath, amsthm}
\usepackage{amssymb}
\usepackage{hyperref}
\usepackage{multicol}
\usepackage{xcolor}
\usepackage{graphicx}
\usepackage{wrapfig}

\newcommand\letsymbol{\mathord{\sqsupset}}
\usepackage[russian]{babel}
\renewcommand\qedsymbol{$\blacktriangleright$}
\newtheorem{theorem}{Теорема}[section]
\newtheorem{lemma}{Лемма}[section]
\theoremstyle{definition}
\newtheorem*{example}{Пример}
\newtheorem*{definition}{Определение}
\newtheorem*{statement}{Утверждение}
\theoremstyle{remark}
\newtheorem*{remark}{Замечание}

\setlength{\topmargin}{-0.5in}
\setlength{\oddsidemargin}{-0.5in}
\textwidth 185mm
\textheight 250mm

\begin{document}
\section{I}
\begin{definition}[Задача линейного программирования]
	Задачей ЛП называется задача поиска максимума или минимума линейной функции
	на множестве, которое описывается линейными ограничениями (равенствами и/или неравенствами)

\end{definition}
\begin{definition}[Общая задача ЛП]
	$    \begin{cases}
			f(x) = c_0 + \sum_{j = 1}^n c_j x_j \to \max (\min)   \\
			\sum_{j = 1}^{n} a_{ij}x_j \# b_i, \, i = 1, \dots, m \\
			x_j \geq 0, j \in J\subseteq \{1, \dots, n\}
		\end{cases}$, где \(x = (x_1, ..., x_n)\in \mathbb{R}^n\) -  вектор переменных

	Матричная запись:

	$\begin{cases}
			f(x) = (c, x) \to \max(\min) \\
			Ax \# b                      \\
			x_j \geq 0, j \in J\subseteq \{1, \dots, n\}
		\end{cases}$, $x = \begin{pmatrix}
			x_1 \\ \vdots \\ x_n
		\end{pmatrix}, b = \begin{pmatrix}
			b_1 \\ \vdots \\ b_m
		\end{pmatrix},
		A = \begin{pmatrix}
			a_{11} & \dots  & a_{1n} \\
			\vdots & \ddots & \vdots \\
			a_{m1} & \dots  & a_{mn}
		\end{pmatrix}$
\end{definition}
\begin{definition}[Допустимое решение задачи ЛП]
	вектор $x\in R^n$, удовлетворяющий ограничениям задачи, называется допустимым решением задачи ЛП
\end{definition}
\begin{definition}[Оптимальное решение задачи ЛП]
	Допустимое решение $x^*\in D$ задачи ЛП называется оптимальным решением, если $f(x) \leq f(x^*) \, \forall x\in D$ в случае задачи максимизации и $f(x) \geq f(x^*) \, \forall x\in D$ в случае задачи минимизации
\end{definition}
\begin{definition}[Разрешимая задача ЛП]
	Задача ЛП называется разрешимой, если она имеет оптимальное решение.
\end{definition}
\begin{definition}[Неразрешимая задача ЛП]
	Задача ЛП называется разрешимой, если она не имеет оптимального решения.
\end{definition}
\begin{definition}[Каноническая задача ЛП]
	$\begin{cases}
			f(x) = c_0 + \sum_{j = 1}^n c_j x_j \to \max        \\
			\sum_{j = 1}^{n} a_{ij}x_j =b_i, \, i = 1, \dots, m \\
			x_j \geq 0, j = 1, \dots, n
		\end{cases}$
\end{definition}
\begin{definition}[Стандартная (симметричная) форма]
	$\begin{cases}
			f(x) = c_0 + \sum_{j = 1}^n c_j x_j \to \max     \\
			\sum_{j = 1}^{n} a_{ij}x_j \leq b_i, \, i = 1, \dots, m \\
			x_j \geq 0, j = 1, \dots, n
		\end{cases}$
\end{definition}
\begin{definition}[Эквивалентные ЗЛП (ЗМП)]
	Две задачи ЛП $P_1, P_2$ называются \textit{эквивалентными}, если любому допустимому решению задачи $P_1$ соответствует некоторое допустимое решение задачи $P_2$ и наоборот, причем оптимальному решению одной задачи соответствует оптимальное решение другой задачи.
\end{definition}
\begin{theorem}[Первая теорема эквивалентности]
	Для любой ЗЛП существует эквивалентная ей каноническая ЗЛП.
\end{theorem}
\begin{theorem}[Вторая теорема эквивалентности]
	Для любой ЗЛП существует эквивалентная ей симметрическая ЗЛП.
\end{theorem}
\begin{definition}[Система с базисом]
	СЛАУ - СЛАУ с базисом, если в каждом уравнении имеется переменная с коэффициентом +1, отсутствующая в других уравнениях. Такие переменные будем называть базисными, остальные не базисными
\end{definition}
\begin{definition}[ПЗЛП]
	КЗЛП называется приведенной, если
	\begin{enumerate}
		\item СЛАУ $Ax = B$ является системой с базисом
		\item Целевая функция выражена через небазисные переменные
	\end{enumerate}
\end{definition}
\begin{definition}[Базисное решение]
	Пусть $\overline{x}$ - решение $Ax = B$. Тогда вектор $\overline{x}$ называется базисным решением СЛАУ, если система вектор-столбцов матрицы А, соответствующая ненулевым компонентам вектора $\overline{x}$, ЛНЗ
\end{definition}\begin{remark}
	Если система однородная, то x = $\overline{0}$ - базисное решение
\end{remark}
\begin{definition}[Базисное решение КЗЛП]
	Неотрицательное базисное решение СЛУ называется базисным решением канонической задачи ЛП
\end{definition}

\begin{definition}[Прямо допустимая симплексная таблица]
	СТ называется прямо допустимой, если $a_{i0}\geq 0, i = 1, \dots, m$ (bшки)
\end{definition}
\begin{definition}[Двойственно допустимая симплексная таблица]
	СТ называется двойственно допустимой, если $a_{0j}\geq 0, i = 1, \dots, n+m$ (cшки)
\end{definition}
\begin{definition}[Проверка на оптимальность в симплекс-методе]
	Если $a_{0j} \ge 0$ для любого $j=1,…,n+m$, то конец - базисное решение x, соответствующее
	симплексной таблице, оптимально.
\end{definition}
\begin{definition}[Проверка на неразрешимость в симплекс-методе]
	Если существует столбец с номером $q\in \{1,...,n+m\}$ такой, что $a_{0q}<0$, и $a_{iq}<0, i=1,…,m$, то
	конец - задача ЛП неразрешима
\end{definition}
\begin{definition}[Выбор ведущего столбца в симплекс-методе]
	Столбец с номером $q\in\{1,...,n+m\}$ выбирается ведущим, если $a_{0q}<0$. Если таких столбцов
	несколько, то выбирается любой из них.
\end{definition}
\begin{definition}[Выбор ведущей строки в симплекс-методе]
	Строка с номером $p\in\{1,...,m\}$ выбирается ведущей, в соответствии с минимальным
	ключевым отношением
	\[\frac{a_{p0}}{a_{pq}} = \min_{a_{iq}>0}\frac{a_{i0}}{a_{iq}}\]
\end{definition}
\begin{definition}[Правило прямоугольника]
	\[a_{ij}' = a_{ij} - \frac{a_{iq}}{a_{pq}}a_{pj}, i = 0, \dots, m, i\neq p, j = 0, \dots, n+m\]
	(в полной симплексной таблице)
\end{definition}
\begin{definition}[Вспомогательная задача ЛП (в методе искусственного базиса)]
	\[h(x, t) = -\sum_{i = 1}^m t_i\to \max\]
	\[a_{11}x_1 + ... + a_{1n}x_n + t_1 = b_1\]
	\[a_{21}x_1 + ... + a_{2n}x_n + t_2 = b_2\]
	\[\dots\]
	\[a_{m1}x_1 + ... + a_{mn}x_n + t_m = b_m\]
	\[x_j\ge 0, j = 1, ..., n, t_i\ge 0, i = 1, \dots, m\]
\end{definition}
\begin{theorem}[Критерий разрешимости]
	Если целевая функция задачи ЛП ограничена сверху (снизу) на непустой множестве
	допустимых решений, то задача максимизации (минимизации) имеет оптимальное
	решение
\end{theorem}
\begin{definition}[Двойственная задача]
	Для ЗЛП I двойственной задачей II является ЗЛП вида:
	$$f(x) = \sum_{j = 1}^n c_j x_j \to \max \leftrightarrow g(y) = \sum_{i = 1}^m b_i y_i\to \min,$$
	$$\sum_{j = 1}^n a_{ij} x_j \leq b_i, i = 1, \dots, l \leftrightarrow y_i\geq 0, i = 1...l,$$
	$$\sum_{j = 1}^n a_{ij} x_j = b_i, i = l+1, \dots m \leftrightarrow y_i\in \mathbb{R}, i = l+1, \dots, m,$$
	$$x_j\geq 0, i = 1, \dots p\leftrightarrow \sum_{i = 1}^m a_{ij} y_i \geq c_j, j = 1, \dots, p$$
	$$x_j \in \mathbb{R}, j = p+1, \dots n \leftrightarrow \sum_{i = 1}^m a_{ij} y_i = c_j, j = p+1, \dots, n$$
	Задачу I называют прямой, а II - двойственной. Стрелки соответствуют сопряженным ограничениям
\end{definition}
\begin{theorem}[Первая теорема двойственности]
	Если одна из пары двойственных задач разрешима, то разрешима и другая, причем оптимальное значение целевых функций совпадает, т.е $f(x^*) = g(y^*)$, где $x^*, y^*$ - оптимальные решения задач I, II соответственно
\end{theorem}
\begin{theorem}[Первый критерий оптимальности]
	Вектор $x^* \in D_I$ является оптимальным решением задачи
	I $\Leftrightarrow \exists y^* \in D_{II}$ т.ч  $g(y^*) = f(x^*)$
\end{theorem}
\begin{definition}[Условия дополняющей нежесткости]
	Будем говорить, что $x\in D_I, y \in D_{II}$ удовлетворяют УДН, если при подстановке в любую пару сопряженных неравенств хотя бы одно из них обращается в равенство. Это означает, что следующие характеристические произведения обращаются в 0:
	\[(\sum_{j = 1}^n a_{ij}x_j - b_i)y_i = 0, i  = 1, \dots m\]
	\[x_j (\sum_{i = 1}^m a_{ij}y_i - c_j) = 0, j = 1, \dots n\]
\end{definition}
\begin{theorem}[Вторая теорема двойственности]
	$x^* \in D_I, y^*\in D_{II}.$ оптимальны в задачах I, II тогда и только тогда, когда они удовлетворяют УДН.
\end{theorem}
\begin{theorem}[Второй критерий оптимальности (следствие)]
	\(x^* \in D_I\) является оптимальным решением I \(\Leftrightarrow\)
	\(\exists y^* \in D_{II}\) т.ч. \(x^*\) и \(y^*\) удовлетворяют УДН
\end{theorem}
\begin{definition}[Малое (допустимое) изменение]
	Малое (допустимое) изменение ресурса P1 - такое изменение $\Delta b_1 = b_1' - b_1$ для кот в задаче I' существует оптимальное решение той же структуры, что и оптимальное решение исходной задачи I
\end{definition}
\begin{definition}[3-я теорема двойственности]
	При допустимом изменении i-того ресурса приращение целевой функции прямо пропорционально изменению ресурса с коэффициентом пропорциональности, равным $y_i^*$

	$$\Delta_i F = \Delta b_i y_i^*, \Delta_i F = F(b_1, \dots, b_{i-1}, b_i + \Delta b_i, \dots, b_m)-F(b_1, \dots, b_{i-1}, b_i, \dots, b_m)$$
\end{definition}
\section{II}
\begin{definition}[Выпуклое множество]
	Множество называется выпуклым, если вместе с двумя его точками оно содержит отрезок, их соединяющий, или
	\[\forall x^1, x^2 \in D \quad \forall \lambda \in (0, 1) \quad
		x^* = (1 - \lambda)x^1 + \lambda x^2 \in D\]
\end{definition}
\begin{definition}[Выпуклая функция]
	Функция $f:D\to R$ (D - выпкуло) называется выпуклой, если
	\[\forall x^1, x^2 \in D, \forall \lambda \in (0, 1) \quad f((1-\lambda)x^1 +\lambda x^2) \le
		(1-\lambda)f(x^1) + \lambda f(x^2)\]
\end{definition}\begin{definition}[Задача ВП]:
	\begin{center}
		\(f(x) \to \min\) \\
		\(\phi_i(x) \le 0, i = 1, \dots, m\)\\
		\(x \in G\)
	\end{center}
	Здесь $\phi_i, f$ - выпуклые в G функции,
	G - выпуклое замкнутое множество ($\mathbb{R}^n,\mathbb{R}_+^n $)

\end{definition}
\begin{definition}[Условие Слейтера]
	\textbf{(УС)}
	\[\exists \overline{x}\in G, \phi_i(\overline{x})<0.\]
	\[D = \{x\in G|\phi_i \le 0, i = 1, \dots, m\} - \textbf{множество допустимых решений задачи ВП.}\]
	УС гарантирует существование внутренних точек множества D.
\end{definition}
\begin{theorem}[О градиенте и производной по направлению]
	Если $f(x)$ дифференцируема в точке $x^0$, то
	предел
	\[\lim_{\lambda \to 0+0} \frac{f(x^0+\lambda z) - f(x^0)}{\lambda},\] существует и равен
	\[f_z'(x^0) = (\nabla f(x^0), z)\]
\end{theorem}

Пусть задана точка $x_0\in D$.
$I_0 = \{i \; | \; \phi_i(x^0) = 0\}$ - множество индексов \textit{активных} ограничений

\begin{definition}[Возможное направление]
	Направление z называется возможным (допустимым) в $x^0$, если $(\nabla \phi_i(x^0), z)<0 \quad \forall i \in I_0$
\end{definition}
\begin{definition}[Прогрессивное направление]
	Направление z называется прогрессивным в точке $x^0$, если
	\[\begin{cases}
			(\nabla \phi_i(x^0), z)<0 \quad \forall i \in I_0 \\
			(\nabla f(x^0), z)<0
		\end{cases}\]
\end{definition}
\begin{theorem}[Критерий оптимальности ЗВП]
	$x^* \in D$ - оптимальное решение задачи ВП $\Leftrightarrow$
	в точке $x^*$ нет прогрессивного направления, т.е не существует $z\in R^n$:
	\[\begin{cases}
			(\nabla \phi_i(x^0), z)<0 \quad \forall i \in I_0 \\
			(\nabla f(x^0), z)<0
		\end{cases}\]
\end{theorem}
\begin{definition}[Каноническая ЗВП]
	Канонической задачей ВП называется задача ВП с линейной целевой функцией, т.е $f(x) = (c, x)\to \min$
\end{definition}
\begin{theorem}[Теорема Куна-Таккера о седловой точке]
	$x^*\in G$ - оптимальное решение задачи выпуклого программирования тогда и только тогда, когда существует $y^* \ge 0$, такое что $(x^*, y^*)$ является седловой точкой функции Лагранжа
\end{theorem}
\subsubsection*{Дифференциальная форма 1}
Рассмотрим задачу ВП  (I)
\[f(x)\to \min\]
\[\phi_i(x)\le 0, i = 1, ..., m\]
\[x_j \ge 0, j = 1, ..., n\]
\[G = \{x\in R^n, x_j \ge 0, j = 1, ..., n\}\]
$f, \phi_i$ - непрерывно дифференцируемые на G

Выполнено УС
\[L(x, y) = f(x) +\sum_{i = 1}^{m}y_i \phi_i(x), y_i\ge 0 , i = 1, \dots, m\]
Обозначим \[\nabla_x L_x(x^*, y^*) = (\frac{\partial L}{\partial x_1}, ..., \frac{\partial L}{\partial x_n})|_{(x^*, y^*)}\]
\[\nabla_x L_y(x^*, y^*) = (\frac{\partial L}{\partial y_1}, ..., \frac{\partial L}{\partial y_n})|_{(x^*, y^*)} = (\phi_1(x^*), \dots, \phi_m(x^*))\]

\begin{theorem}[Куна-Таккера в дифференциальной форме 1]
	Точка $x^* \ge 0$ является оптимальным решением задачи (I) тогда  и только тогда, когда существует $y^*\ge 0$ такой, что выполняются следующие условия:
	\begin{enumerate}
		\item $\nabla_x L(x^*, y^*) \ge 0,$ т.е. $\frac{\partial L(x, y^*)}{\partial x_j} |_{x = x^*} \ge 0, \forall j = 1, \dots, n$
		\item $(x^*, \nabla_x L(x^*, y^*)) = 0$ т.е $\sum_{i = 1}^n x^*_j\frac{\partial L(x, y^*)}{\partial x_j}|_{x^*} = 0$
		\item $\nabla_y L(x^*, y^*) \le 0,$ т.е $\phi_i(x^*)\le 0, i = 1, ..., m$
		\item $(y^*, \nabla_y L(x^*, y^*)) = 0, $ т.е $\sum_{i = 1}^{m}y^*_i \phi_i(x^*) = 0$
	\end{enumerate}
\end{theorem}
\subsection*{Дифференциальная форма 2}
Рассмотрим задачу ВП (II)
\[f(x)\to \min\]
\[\phi_i(x)\le 0, i = 1, ..., m\]
\[x\in R^n\]
\[G = R^n\]
$f, \phi_i$ - непрерывно дифференцируемые на G
\begin{theorem}[Куна-Таккера в дифференциальной форме 2]
	Точка $x^*\in R^n$ является оптимальной точкой задачи (II)
	тогда и только тогда, когда существует $y^*\ge 0$ такое, что выполняются условия
	\begin{enumerate}
		\item $\nabla_x L(x^*, y^*) = 0,$ т.е.
		      $\frac{\partial L(x, y^*)}{\partial x_j} |_{x = x^*} = 0, \forall j = 1, \dots, n$
		\item $\nabla_y L(x^*, y^*) \le 0,$ т.е $\phi_i(x^*)\le 0, i = 1, ..., m$
		\item $(y^*, \nabla_y L(x^*, y^*)) = 0, $ т.е $\sum_{i = 1}^{m}y^*_i \phi_i(x^*) = 0$, или $\forall i: y^*_i \phi_i(x^*) = 0$
	\end{enumerate}
\end{theorem}
\begin{definition}[Задача ЦЛП]
	\begin{equation}
		f(x) = \sum_{j=1}^n c_j x_j \to \max
	\end{equation}
	\begin{equation}
		\sum_{j = 1}^n a_{ij}x_j \# b_i, i = 1, \dots, m
	\end{equation}
	\begin{equation}
		x_j \ge 0, j = 1, \dots, n
	\end{equation}
	\begin{equation}
		x_j \in \mathbb{Z}, j =1, \dots, n
	\end{equation}

	\(c_j, b_i, a_{ij} \in \mathbb{Z} \text{ или } \mathbb{Q}\)
\end{definition}

\begin{definition}[Правильное отсечение]
	Доп. линейное ограничение - правильное отсечение, если
	\begin{enumerate}
		\item оно отсекает часть области D, содержащее нецелочисленное оптимальное решение $x^0$ текущей задачи ЛП.
		\item В отсекаемой части области не должно быть ни одного допустимого решения задачи ЦЛП (ограничение сохраняет все допустимые целочисленные решения)
	\end{enumerate}
\end{definition}
\begin{definition}[Отсечение Гомори]
	Имеем оптимальную с-таблицу $a_{ij, i = 0, \dots, m, j = 0, \dots, n}$

	Рассмотрим $a_{l0}\notin \mathbb{Z}$. l выбираем с дробной частью по правилу "первая сверху" ($l\in \{0, \dots, n\}$)

	Отсечение Гомори - дополнительное линейное ограничение
	\[\sum_{j\in Nb} \{a_{lj}\}x_j \geq \{a_{l0}\}\], где Nb - множество индексов небазисных переменных, $\{x\}$ - дробная часть x
\end{definition}
\section{III}
\begin{definition}[Квадратичная форма]
	Квадратичной формой от n переменных называется функция вида
	\[g(x)  = (cx, x) = \sum_{i=1}^{n}\sum_{j=1}^{n}c_{ij}x_i x_j\], где C - симметричная матрица, $c_{ij} = c_{ji}$

	Диагональные элементы - коэффициенты при квадратах переменных, а остальные - половине коэффициентов
\end{definition}
\begin{definition}[Квадратичная функция]
	Функция вида $f(x) = \sum_{i = 1}^n \sum _{j = 1}^nc_{ij}x_i x_j +\sum_{i = 1}^n p_i x + p_0 = (Cx, x) + (p, x) + p_0, p \in R^n, p_0\in R$
\end{definition}
\begin{definition}[ЗКП]
	Задача $f(x) = \sum_{i=1}^{n}\sum_{j=1}^{n} c_{ij}x_i x_j +\sum_{j  = 1}^{n}p_j x_j +p_0\to \min$
	\[\sum_{j = 1}^n a_{ij}x_j \le b_i, i = 1, \dots, m\]
	\[x_j\ge 0, j = 1, \dots, n\]
	называется задачей квадратичного программирования
\end{definition}
\begin{theorem}[Критерий оптимальности для задачи квадратичного программирования]
	Вектор $x^*\ge 0$ является оптимальным решением задачи КП $\Leftrightarrow \; \exists$ неотрицательные векторы $y^*, u^*\in R^m$и $v^*\in R^n$ что выполняется
	\begin{enumerate}
		\item $2Cx^* + A^T y^* - v^* = -p$
		\item $Ax^* + u^* = b$
		\item $(x^*, v^*) = 0$
		\item $(y^*, u^*) = 0$
	\end{enumerate}
	\begin{theorem}
		Если $\exists$ неотрицательное решение системы 1-2, уд. 3-4, то существует и базисное неотрицательное решение этой же системы, удовлетворяющее тем же свойствам
	\end{theorem}
\end{theorem}
\begin{definition}[Многокритериальная задача]
	\[\begin{cases}
			f_i(x)\to opt i = 1, \dots, p \\
			x\in D
		\end{cases}\]
	Многокритериальная задача - задача МП, в которой не один, а несколько критериев (целевых функций)
\end{definition}
\begin{definition}[Парето-оптимальное решение]
	\[\begin{cases}
			f_i(x)\to \max i = 1, \dots, p \\
			x\in D
		\end{cases}\]
	$x^*\in D$ называется Парето-оптимальным решением, если
	не существует точки $x\in D$, такой что
	\[f_k(x) \ge f_k(x^*), k = 1, \dots, p,\]
	причем хотя бы для одного k неравенство строгое
\end{definition}
\begin{definition}
	Полное множество альтернатив (ПМА) - минимальное по мощности подмножество $X^0 \subseteq X^*$, т.ч. образ
	\[f(X^0) = f(X^*)\]
\end{definition}
\begin{definition}[Линейная свертка]
	Линейная свертка - функция вида
	\[f = \sum_{i=1}^{}\lambda_i f_i, \lambda_i -\text{ относительный показатель важности критерия}\]
	Как правило, $\lambda_i$ определяется с использованием экспертных оценок.
\end{definition}
\end{document}