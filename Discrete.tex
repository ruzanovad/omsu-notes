\documentclass[a4paper]{article}
\usepackage[utf8]{inputenc}
\usepackage[T2A]{fontenc}
\usepackage{amsfonts}
\usepackage{amsmath, amsthm}
\usepackage{amssymb}
\usepackage{hyperref}
\usepackage{multicol}

\newcommand\letsymbol{\mathord{\sqsupset}}
\usepackage[russian]{babel}
\renewcommand\qedsymbol{$\blacktriangleright$}
\newtheorem{theorem}{Теорема}[section]
\newtheorem{lemma}{Лемма}[section]
\theoremstyle{definition}
\newtheorem*{example}{Пример}
\newtheorem*{definition}{Определение}
\theoremstyle{remark}
\newtheorem*{remark}{Замечание}

\setlength{\topmargin}{-0.5in}
\setlength{\oddsidemargin}{-0.5in}
\textwidth 185mm
\textheight 250mm

\begin{document}
% \begin{multicols*}{2}
    \tableofcontents
    \pagenumbering{arabic}
    \setcounter{page}{1}
    \section{Теория булевых функций}
    \subsection{Определение булевой функции (БФ). Количество БФ от n переменных. Таблица истинности БФ}
    \begin{definition}
        Булева функция от n переменных - это отображение $\{0,1\}^n \rightarrow \{0, 1\}$
    \end{definition}

    \begin{remark}
        Количество БФ от n переменных - $2^{2^n}$
    \end{remark}
    \begin{proof}
        Каждая булева функция определяется своим столбцом значений.
         Столбец является булевым вектором длины $m=2n$, где n – число аргументов функции.
          Число различных векторов длины m (а значит и число булевых функций, зависящих от n переменных) равно $2^m=2^{2^n}$
    \end{proof}
    \subsection{Булевы функции одной и двух переменных (их таблицы, названия)}
Булевы функции одной переменной:
         \begin{tabular}{c|cccc}
        x & $f_1$ & $f_2$ & $f_3$ & $f_4$ \\
        \hline
        0 & 0 & 0 & 1 & 1 \\
        1 & 0 & 1 & 0 & 1 \\ 
        \end{tabular}
    $f_1$ - тождественный 0, $f_2$ - тождественная функция, $f_3$ - отрицание ($\neg$), $f_4$ - тождественная 1
\\ \\
    Булевы функции двух переменных
    \begin{tabular}{cc|cccccccccccccccc}
        x & y & 0 & $\wedge$ & $\rightarrow '$ & $x$ & $\leftarrow '$ & $y$ & $+$ & $\vee$ & $\downarrow$ & $\leftrightarrow$ & $y'$ & $\leftarrow$ & $x'$ & $\rightarrow$ & | & 1\\
        \hline
        0 & 0 & 0 & 0 & 0 & 0 & 0 & 0 & 0 & 0 & 1 & 1 & 1 & 1 & 1 & 1 & 1 & 1 \\
        0 & 1 & 0 & 0 & 0 & 0 & 1 & 1 & 1 & 1 & 0 & 0 & 0 & 0 & 1 & 1 & 1 & 1 \\ 
        1 & 0 & 0 & 0 & 1 & 1 & 0 & 0 & 1 & 1 & 0 & 0 & 1 & 1 & 0 & 0 & 1 & 1 \\
        1 & 1 & 0 & 1 & 0 & 1 & 0 & 1 & 0 & 1 & 0 & 1 & 0 & 1 & 0 & 1 & 0 & 1 \\
        \end{tabular}
    \begin{enumerate}
        \item $\wedge$ - конъюнкция
        \item $\leftarrow$ -  антиимпликация
        \item $\rightarrow$ - импликация
        \item $\vee$ - дизъюнкция
        \item | - штрих Шеффера
        \item $\downarrow$ - стрелка Пирса
        \item + -  взаимоисключающее или, сложение по модулю 2 (XOR)
    \end{enumerate}
    \subsection{Формулы логики высказываний. Представление БФ формулами}
    \begin{definition}
        Формула логики высказываний - слово алфавита логики высказываний,
        построенное по следующим правилам:
        \begin{enumerate}
            \item символ переменной - формула
            \item символы 0 и 1 - формулы
            \item если $\Phi_1$ и $\Phi_2$ - формулы, то слова ($\Phi_1 \& \Phi_2$),
            ($\Phi_1 \leftrightarrow \Phi_2$), ($\Phi_1 \rightarrow \Phi_2$), 
            ($\Phi_1 | \Phi_2$), $\dots$ , $\Phi_1 '$ тоже формулы
        \end{enumerate}
    \end{definition}
% \end{multicols*}

\end{document}