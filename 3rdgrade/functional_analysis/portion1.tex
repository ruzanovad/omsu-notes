\documentclass[a4paper, 12pt]{article}
\usepackage[utf8]{inputenc}
\usepackage[T2A]{fontenc}
\usepackage{amsfonts}
\usepackage{amsmath, amsthm}
\usepackage{amssymb}
\usepackage{hyperref}

\newcommand\letsymbol{\mathord{\sqsupset}}
\usepackage[russian]{babel}
\renewcommand\qedsymbol{$\blacktriangleright$}
\newtheorem{theorem}{Теорема}[section]
\newtheorem{lemma}{Лемма}[section]
\theoremstyle{definition}
\newtheorem*{example}{Пример}
\newtheorem*{definition}{Определение}
\theoremstyle{remark}
\newtheorem*{remark}{Замечание}

\DeclareMathOperator{\Int}{int}
\DeclareMathOperator{\clo}{cl}

\setlength{\topmargin}{-0.5in}
\setlength{\oddsidemargin}{-0.5in}
\textwidth 185mm
\textheight 250mm

\begin{document}
\begin{enumerate}
    \item[9] Привести пример $A\subset \mathbb{R}$, т.ч следующие множества
    попарно различны:
    \[A, \clo A, \Int A, \clo (\Int A), \Int (\clo A), \Int (\clo (\Int A)), \clo(\Int (\clo A))\]

    \[A = [0, \frac12) \cup (\frac12, 1] \cup \{ 2\} \cup ([3, 4] \cap \mathbb{Q})\]
    \[\Int A = (0, \frac12) \cup (\frac12, 1)\]
    \[\clo A = [0,1] \cup \{2\} \cup [3, 4]\]
    \[\Int (\clo A) = (0, 1) \cup (3, 4)\]
    \[\clo (\Int A) = [0, 1]\]
    \[\clo (\Int (\clo A)) = [0,1] \cup [3, 4]\]
    \[\Int(\clo(\Int A)) = (0, 1)\]
    \item[10]
    Какое наибольшее число попарно различных множеств можно получить
    из подмножества топологического пространства, последовательно
    применяя к нему операции замыкания и внутренности?

    7, т.к из 9 номера мы получили 7 различных множеств
    и известно, что 
    \begin{equation}
        \clo (\Int (\clo (\Int A)))= \clo (\Int A)
    \end{equation}
    \begin{equation}
        \Int (\clo (\Int (\clo A)))= \Int (\clo A)
    \end{equation}
    % \[ (1)\]
    % \[ (2)\]
    \begin{proof}
        \begin{enumerate}
            \item[1] $\Int A \subset A \implies \clo (\Int A)\subset \clo A$
            $\implies \clo(\Int(\clo A))\subset \clo (\clo A) = \clo A\implies$

            $\clo (\Int (\clo (\Int A))) \subset \clo (\Int A)$

            $\Int A \subset \clo (\Int A) \implies \Int A \subset \Int (\clo (\Int A)) \implies$
            $\clo (\Int A) \subset \clo (\Int (\clo (\Int A)))$

            \item[2]  $\Int A \subset A \implies \Int (\clo A) \subset \clo A \implies \clo(\Int(\clo A)) \subset \clo A$
            $\Int(\clo (\Int (\clo A)))\subset \Int(\clo A)$

            $A\subset \clo A \implies \Int (\clo A) \subset \clo (\Int (\clo A)) \implies$
            $\Int (\clo A) \subset \Int (\clo (\Int (\clo A)))$
        \end{enumerate}
    \end{proof}

\end{enumerate}


\end{document}